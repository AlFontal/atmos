% Generated by Sphinx.
\def\sphinxdocclass{report}
\documentclass[letterpaper,10pt,english]{sphinxmanual}
\usepackage[utf8]{inputenc}
\DeclareUnicodeCharacter{00A0}{\nobreakspace}
\usepackage{cmap}
\usepackage[T1]{fontenc}
\usepackage{babel}
\usepackage{times}
\usepackage[Bjarne]{fncychap}
\usepackage{longtable}
\usepackage{sphinx}
\usepackage{multirow}


\title{atmos Documentation}
\date{May 29, 2015}
\release{0.2.1}
\author{Jeremy McGibbon}
\newcommand{\sphinxlogo}{}
\renewcommand{\releasename}{Release}
\makeindex

\makeatletter
\def\PYG@reset{\let\PYG@it=\relax \let\PYG@bf=\relax%
    \let\PYG@ul=\relax \let\PYG@tc=\relax%
    \let\PYG@bc=\relax \let\PYG@ff=\relax}
\def\PYG@tok#1{\csname PYG@tok@#1\endcsname}
\def\PYG@toks#1+{\ifx\relax#1\empty\else%
    \PYG@tok{#1}\expandafter\PYG@toks\fi}
\def\PYG@do#1{\PYG@bc{\PYG@tc{\PYG@ul{%
    \PYG@it{\PYG@bf{\PYG@ff{#1}}}}}}}
\def\PYG#1#2{\PYG@reset\PYG@toks#1+\relax+\PYG@do{#2}}

\expandafter\def\csname PYG@tok@gd\endcsname{\def\PYG@tc##1{\textcolor[rgb]{0.63,0.00,0.00}{##1}}}
\expandafter\def\csname PYG@tok@gu\endcsname{\let\PYG@bf=\textbf\def\PYG@tc##1{\textcolor[rgb]{0.50,0.00,0.50}{##1}}}
\expandafter\def\csname PYG@tok@gt\endcsname{\def\PYG@tc##1{\textcolor[rgb]{0.00,0.27,0.87}{##1}}}
\expandafter\def\csname PYG@tok@gs\endcsname{\let\PYG@bf=\textbf}
\expandafter\def\csname PYG@tok@gr\endcsname{\def\PYG@tc##1{\textcolor[rgb]{1.00,0.00,0.00}{##1}}}
\expandafter\def\csname PYG@tok@cm\endcsname{\let\PYG@it=\textit\def\PYG@tc##1{\textcolor[rgb]{0.25,0.50,0.56}{##1}}}
\expandafter\def\csname PYG@tok@vg\endcsname{\def\PYG@tc##1{\textcolor[rgb]{0.73,0.38,0.84}{##1}}}
\expandafter\def\csname PYG@tok@m\endcsname{\def\PYG@tc##1{\textcolor[rgb]{0.13,0.50,0.31}{##1}}}
\expandafter\def\csname PYG@tok@mh\endcsname{\def\PYG@tc##1{\textcolor[rgb]{0.13,0.50,0.31}{##1}}}
\expandafter\def\csname PYG@tok@cs\endcsname{\def\PYG@tc##1{\textcolor[rgb]{0.25,0.50,0.56}{##1}}\def\PYG@bc##1{\setlength{\fboxsep}{0pt}\colorbox[rgb]{1.00,0.94,0.94}{\strut ##1}}}
\expandafter\def\csname PYG@tok@ge\endcsname{\let\PYG@it=\textit}
\expandafter\def\csname PYG@tok@vc\endcsname{\def\PYG@tc##1{\textcolor[rgb]{0.73,0.38,0.84}{##1}}}
\expandafter\def\csname PYG@tok@il\endcsname{\def\PYG@tc##1{\textcolor[rgb]{0.13,0.50,0.31}{##1}}}
\expandafter\def\csname PYG@tok@go\endcsname{\def\PYG@tc##1{\textcolor[rgb]{0.20,0.20,0.20}{##1}}}
\expandafter\def\csname PYG@tok@cp\endcsname{\def\PYG@tc##1{\textcolor[rgb]{0.00,0.44,0.13}{##1}}}
\expandafter\def\csname PYG@tok@gi\endcsname{\def\PYG@tc##1{\textcolor[rgb]{0.00,0.63,0.00}{##1}}}
\expandafter\def\csname PYG@tok@gh\endcsname{\let\PYG@bf=\textbf\def\PYG@tc##1{\textcolor[rgb]{0.00,0.00,0.50}{##1}}}
\expandafter\def\csname PYG@tok@ni\endcsname{\let\PYG@bf=\textbf\def\PYG@tc##1{\textcolor[rgb]{0.84,0.33,0.22}{##1}}}
\expandafter\def\csname PYG@tok@nl\endcsname{\let\PYG@bf=\textbf\def\PYG@tc##1{\textcolor[rgb]{0.00,0.13,0.44}{##1}}}
\expandafter\def\csname PYG@tok@nn\endcsname{\let\PYG@bf=\textbf\def\PYG@tc##1{\textcolor[rgb]{0.05,0.52,0.71}{##1}}}
\expandafter\def\csname PYG@tok@no\endcsname{\def\PYG@tc##1{\textcolor[rgb]{0.38,0.68,0.84}{##1}}}
\expandafter\def\csname PYG@tok@na\endcsname{\def\PYG@tc##1{\textcolor[rgb]{0.25,0.44,0.63}{##1}}}
\expandafter\def\csname PYG@tok@nb\endcsname{\def\PYG@tc##1{\textcolor[rgb]{0.00,0.44,0.13}{##1}}}
\expandafter\def\csname PYG@tok@nc\endcsname{\let\PYG@bf=\textbf\def\PYG@tc##1{\textcolor[rgb]{0.05,0.52,0.71}{##1}}}
\expandafter\def\csname PYG@tok@nd\endcsname{\let\PYG@bf=\textbf\def\PYG@tc##1{\textcolor[rgb]{0.33,0.33,0.33}{##1}}}
\expandafter\def\csname PYG@tok@ne\endcsname{\def\PYG@tc##1{\textcolor[rgb]{0.00,0.44,0.13}{##1}}}
\expandafter\def\csname PYG@tok@nf\endcsname{\def\PYG@tc##1{\textcolor[rgb]{0.02,0.16,0.49}{##1}}}
\expandafter\def\csname PYG@tok@si\endcsname{\let\PYG@it=\textit\def\PYG@tc##1{\textcolor[rgb]{0.44,0.63,0.82}{##1}}}
\expandafter\def\csname PYG@tok@s2\endcsname{\def\PYG@tc##1{\textcolor[rgb]{0.25,0.44,0.63}{##1}}}
\expandafter\def\csname PYG@tok@vi\endcsname{\def\PYG@tc##1{\textcolor[rgb]{0.73,0.38,0.84}{##1}}}
\expandafter\def\csname PYG@tok@nt\endcsname{\let\PYG@bf=\textbf\def\PYG@tc##1{\textcolor[rgb]{0.02,0.16,0.45}{##1}}}
\expandafter\def\csname PYG@tok@nv\endcsname{\def\PYG@tc##1{\textcolor[rgb]{0.73,0.38,0.84}{##1}}}
\expandafter\def\csname PYG@tok@s1\endcsname{\def\PYG@tc##1{\textcolor[rgb]{0.25,0.44,0.63}{##1}}}
\expandafter\def\csname PYG@tok@gp\endcsname{\let\PYG@bf=\textbf\def\PYG@tc##1{\textcolor[rgb]{0.78,0.36,0.04}{##1}}}
\expandafter\def\csname PYG@tok@sh\endcsname{\def\PYG@tc##1{\textcolor[rgb]{0.25,0.44,0.63}{##1}}}
\expandafter\def\csname PYG@tok@ow\endcsname{\let\PYG@bf=\textbf\def\PYG@tc##1{\textcolor[rgb]{0.00,0.44,0.13}{##1}}}
\expandafter\def\csname PYG@tok@sx\endcsname{\def\PYG@tc##1{\textcolor[rgb]{0.78,0.36,0.04}{##1}}}
\expandafter\def\csname PYG@tok@bp\endcsname{\def\PYG@tc##1{\textcolor[rgb]{0.00,0.44,0.13}{##1}}}
\expandafter\def\csname PYG@tok@c1\endcsname{\let\PYG@it=\textit\def\PYG@tc##1{\textcolor[rgb]{0.25,0.50,0.56}{##1}}}
\expandafter\def\csname PYG@tok@kc\endcsname{\let\PYG@bf=\textbf\def\PYG@tc##1{\textcolor[rgb]{0.00,0.44,0.13}{##1}}}
\expandafter\def\csname PYG@tok@c\endcsname{\let\PYG@it=\textit\def\PYG@tc##1{\textcolor[rgb]{0.25,0.50,0.56}{##1}}}
\expandafter\def\csname PYG@tok@mf\endcsname{\def\PYG@tc##1{\textcolor[rgb]{0.13,0.50,0.31}{##1}}}
\expandafter\def\csname PYG@tok@err\endcsname{\def\PYG@bc##1{\setlength{\fboxsep}{0pt}\fcolorbox[rgb]{1.00,0.00,0.00}{1,1,1}{\strut ##1}}}
\expandafter\def\csname PYG@tok@mb\endcsname{\def\PYG@tc##1{\textcolor[rgb]{0.13,0.50,0.31}{##1}}}
\expandafter\def\csname PYG@tok@ss\endcsname{\def\PYG@tc##1{\textcolor[rgb]{0.32,0.47,0.09}{##1}}}
\expandafter\def\csname PYG@tok@sr\endcsname{\def\PYG@tc##1{\textcolor[rgb]{0.14,0.33,0.53}{##1}}}
\expandafter\def\csname PYG@tok@mo\endcsname{\def\PYG@tc##1{\textcolor[rgb]{0.13,0.50,0.31}{##1}}}
\expandafter\def\csname PYG@tok@kd\endcsname{\let\PYG@bf=\textbf\def\PYG@tc##1{\textcolor[rgb]{0.00,0.44,0.13}{##1}}}
\expandafter\def\csname PYG@tok@mi\endcsname{\def\PYG@tc##1{\textcolor[rgb]{0.13,0.50,0.31}{##1}}}
\expandafter\def\csname PYG@tok@kn\endcsname{\let\PYG@bf=\textbf\def\PYG@tc##1{\textcolor[rgb]{0.00,0.44,0.13}{##1}}}
\expandafter\def\csname PYG@tok@o\endcsname{\def\PYG@tc##1{\textcolor[rgb]{0.40,0.40,0.40}{##1}}}
\expandafter\def\csname PYG@tok@kr\endcsname{\let\PYG@bf=\textbf\def\PYG@tc##1{\textcolor[rgb]{0.00,0.44,0.13}{##1}}}
\expandafter\def\csname PYG@tok@s\endcsname{\def\PYG@tc##1{\textcolor[rgb]{0.25,0.44,0.63}{##1}}}
\expandafter\def\csname PYG@tok@kp\endcsname{\def\PYG@tc##1{\textcolor[rgb]{0.00,0.44,0.13}{##1}}}
\expandafter\def\csname PYG@tok@w\endcsname{\def\PYG@tc##1{\textcolor[rgb]{0.73,0.73,0.73}{##1}}}
\expandafter\def\csname PYG@tok@kt\endcsname{\def\PYG@tc##1{\textcolor[rgb]{0.56,0.13,0.00}{##1}}}
\expandafter\def\csname PYG@tok@sc\endcsname{\def\PYG@tc##1{\textcolor[rgb]{0.25,0.44,0.63}{##1}}}
\expandafter\def\csname PYG@tok@sb\endcsname{\def\PYG@tc##1{\textcolor[rgb]{0.25,0.44,0.63}{##1}}}
\expandafter\def\csname PYG@tok@k\endcsname{\let\PYG@bf=\textbf\def\PYG@tc##1{\textcolor[rgb]{0.00,0.44,0.13}{##1}}}
\expandafter\def\csname PYG@tok@se\endcsname{\let\PYG@bf=\textbf\def\PYG@tc##1{\textcolor[rgb]{0.25,0.44,0.63}{##1}}}
\expandafter\def\csname PYG@tok@sd\endcsname{\let\PYG@it=\textit\def\PYG@tc##1{\textcolor[rgb]{0.25,0.44,0.63}{##1}}}

\def\PYGZbs{\char`\\}
\def\PYGZus{\char`\_}
\def\PYGZob{\char`\{}
\def\PYGZcb{\char`\}}
\def\PYGZca{\char`\^}
\def\PYGZam{\char`\&}
\def\PYGZlt{\char`\<}
\def\PYGZgt{\char`\>}
\def\PYGZsh{\char`\#}
\def\PYGZpc{\char`\%}
\def\PYGZdl{\char`\$}
\def\PYGZhy{\char`\-}
\def\PYGZsq{\char`\'}
\def\PYGZdq{\char`\"}
\def\PYGZti{\char`\~}
% for compatibility with earlier versions
\def\PYGZat{@}
\def\PYGZlb{[}
\def\PYGZrb{]}
\makeatother

\renewcommand\PYGZsq{\textquotesingle}

\begin{document}

\maketitle
\tableofcontents
\phantomsection\label{index::doc}


Contents:


\chapter{Introduction to atmos}
\label{intro:welcome-to-atmos-s-documentation}\label{intro::doc}\label{intro:introduction-to-atmos}

\section{What this package does}
\label{intro:what-this-package-does}
\textbf{atmos} is meant to be a library of utility code for use in atmospheric
sciences. Its main functionality is currently to take input variables (like
pressure, virtual temperature, water vapor mixing ratio, etc.) and information
about what assumptions you're willing to make (hydrostatic? low water vapor?
ignore virtual temperature correction? use an empirical formula for
equivalent potential temperature?), and from that calculate any desired
output variables that you request and can be calculated.


\section{Variable names}
\label{intro:variable-names}
To make coding simpler by avoiding long names for quantities, a set of fairly
reasonable short-forms for different quantities are used by this package.
For example, air density is represented by ``rho'', and air temperature by ``T''.
For a complete list of quantities and their abbreviations, see the
documentation for \code{atmos.calculate()} or \code{atmos.FluidSolver}.


\section{Units}
\label{intro:units}
Currently, all quantities are input and output in SI units. Notably, all
pressures are input and output in Pascals, and temperature-like quantities
in degrees Kelvin. A full list of units for different variables is available
in the documentation for \code{atmos.calculate()} or
\code{atmos.FluidSolver}.

There are plans to later allow the use of alternate units by keyword arguments
or by subclassing \code{atmos.FluidSolver}, but this is not currently
implemented.


\section{Assumptions}
\label{intro:assumptions}
By default, a set of (what are hopefully) fairly reasonable assumptions are
used by \code{atmos.FluidSolver} and \code{atmos.calculate()}. These can be
added to or removed from
by tuples of string options supplied as keyword arguments \emph{add\_assumptions}
and \emph{remove\_assumptions}, respectively, or completely overridden by supplying
a tuple for the keyword argument \emph{assumptions}. For information on what
default assumptions are used and all assumptions available, see the
documentation for \code{atmos.calculate()} or \code{atmos.FluidSolver}.


\section{Requests and Feedback}
\label{intro:requests-and-feedback}
This module is in ongoing development, and feedback is appreciated. In
particular, if there is functionality you would like to see or equations
that should be added (or corrected), please e-mail mcgibbon (at) uw \{dot\} edu.


\chapter{Using calculate()}
\label{calculate:using-calculate}\label{calculate::doc}

\section{What does calculate do?}
\label{calculate:what-does-calculate-do}
\code{atmos.calculate()} takes input variables (like
pressure, virtual temperature, water vapor mixing ratio, etc.) and information
about what assumptions you're willing to make (hydrostatic? low water vapor?
ignore virtual temperature correction? use an empirical formula for
equivalent potential temperature?), and from that calculates any desired
output variables that you request and can be calculated.

This function is essentially a wrapper for \code{atmos.FluidSolver}, so
much or all of its functionality will be the same, and the documentation for
the two is very similar.


\section{What can it calculate?}
\label{calculate:what-can-it-calculate}
Anything that can be calculated by equations in {\color{red}\bfseries{}:module:{}`atmos.equations{}`}.
If you find that calculate() can't do a calculation you might expect it
to, check the equations it has available and make sure you're using the right
variables, or enabling the right assumptions. A common problem is using \emph{T}
instead of \emph{Tv} and expecting the ideal gas law to work.


\section{A simple example}
\label{calculate:a-simple-example}
By default, a certain set of assumptions are used, such as that we are
considering an ideal gas, and so can use ideal gas law. This allows us to do
simple calculations that use the default assumptions. For example, to
calculate pressure from virtual temperature and density:

\begin{Verbatim}[commandchars=\\\{\}]
\PYG{g+gp}{\PYGZgt{}\PYGZgt{}\PYGZgt{} }\PYG{k+kn}{import} \PYG{n+nn}{atmos}
\PYG{g+gp}{\PYGZgt{}\PYGZgt{}\PYGZgt{} }\PYG{n}{atmos}\PYG{o}{.}\PYG{n}{calculate}\PYG{p}{(}\PYG{l+s}{\PYGZsq{}}\PYG{l+s}{p}\PYG{l+s}{\PYGZsq{}}\PYG{p}{,} \PYG{n}{Tv}\PYG{o}{=}\PYG{l+m+mf}{273.}\PYG{p}{,} \PYG{n}{rho}\PYG{o}{=}\PYG{l+m+mf}{1.27}\PYG{p}{)}
\PYG{g+go}{99519.638400000011}
\end{Verbatim}

Or to calculate relative humidity from water vapor mixing ratio and
saturation water vapor mixing ratio (which needs no assumptions):

\begin{Verbatim}[commandchars=\\\{\}]
\PYG{g+gp}{\PYGZgt{}\PYGZgt{}\PYGZgt{} }\PYG{k+kn}{import} \PYG{n+nn}{atmos}
\PYG{g+gp}{\PYGZgt{}\PYGZgt{}\PYGZgt{} }\PYG{n}{atmos}\PYG{o}{.}\PYG{n}{calculate}\PYG{p}{(}\PYG{l+s}{\PYGZsq{}}\PYG{l+s}{RH}\PYG{l+s}{\PYGZsq{}}\PYG{p}{,} \PYG{n}{rv}\PYG{o}{=}\PYG{l+m+mf}{0.001}\PYG{p}{,} \PYG{n}{rvs}\PYG{o}{=}\PYG{l+m+mf}{0.002}\PYG{p}{)}
\PYG{g+go}{50.0}
\end{Verbatim}

For a full list of default assumptions, see \code{atmos.calculate()}.


\section{Viewing equation functions used}
\label{calculate:viewing-equation-functions-used}
Calculating pressure from virtual temperature and density, also returning a
list of functions used:

\begin{Verbatim}[commandchars=\\\{\}]
\PYG{g+gp}{\PYGZgt{}\PYGZgt{}\PYGZgt{} }\PYG{k+kn}{import} \PYG{n+nn}{atmos}
\PYG{g+gp}{\PYGZgt{}\PYGZgt{}\PYGZgt{} }\PYG{n}{p}\PYG{p}{,} \PYG{n}{funcs} \PYG{o}{=} \PYG{n}{atmos}\PYG{o}{.}\PYG{n}{calculate}\PYG{p}{(}\PYG{l+s}{\PYGZsq{}}\PYG{l+s}{p}\PYG{l+s}{\PYGZsq{}}\PYG{p}{,} \PYG{n}{Tv}\PYG{o}{=}\PYG{l+m+mf}{273.}\PYG{p}{,} \PYG{n}{rho}\PYG{o}{=}\PYG{l+m+mf}{1.27}\PYG{p}{,} \PYG{n}{debug}\PYG{o}{=}\PYG{n+nb+bp}{True}\PYG{p}{)}
\PYG{g+gp}{\PYGZgt{}\PYGZgt{}\PYGZgt{} }\PYG{n}{funcs}
\PYG{g+go}{(\PYGZlt{}function atmos.equations.p\PYGZus{}from\PYGZus{}rho\PYGZus{}Tv\PYGZus{}ideal\PYGZus{}gas\PYGZgt{},)}
\end{Verbatim}


\section{Adding and removing assumptions}
\label{calculate:adding-and-removing-assumptions}
If you want to use assumptions that are not enabled by default (such as
ignoring the virtual temperature correction), you can use the add\_assumptions
keyword argument, which takes a tuple of strings specifying assumptions.
The exact string to enter for each assumption is detailed in
\code{atmos.calculate()}. For example, to calculate T instead of Tv, neglecting
the virtual temperature correction:

\begin{Verbatim}[commandchars=\\\{\}]
\PYG{g+gp}{\PYGZgt{}\PYGZgt{}\PYGZgt{} }\PYG{k+kn}{import} \PYG{n+nn}{atmos}
\PYG{g+gp}{\PYGZgt{}\PYGZgt{}\PYGZgt{} }\PYG{n}{atmos}\PYG{o}{.}\PYG{n}{calculate}\PYG{p}{(}\PYG{l+s}{\PYGZsq{}}\PYG{l+s}{p}\PYG{l+s}{\PYGZsq{}}\PYG{p}{,} \PYG{n}{T}\PYG{o}{=}\PYG{l+m+mf}{273.}\PYG{p}{,} \PYG{n}{rho}\PYG{o}{=}\PYG{l+m+mf}{1.27}\PYG{p}{,}
\end{Verbatim}
\begin{description}
\item[{add\_assumptions=(`Tv equals T',))}] \leavevmode
99519.638400000011

\end{description}


\section{Overriding assumptions}
\label{calculate:overriding-assumptions}
If you want to ignore the default assumptions entirely, you could specify
your own assumptions:

\begin{Verbatim}[commandchars=\\\{\}]
\PYG{g+gp}{\PYGZgt{}\PYGZgt{}\PYGZgt{} }\PYG{k+kn}{import} \PYG{n+nn}{atmos}
\PYG{g+gp}{\PYGZgt{}\PYGZgt{}\PYGZgt{} }\PYG{n}{assumptions} \PYG{o}{=} \PYG{p}{(}\PYG{l+s}{\PYGZsq{}}\PYG{l+s}{ideal gas}\PYG{l+s}{\PYGZsq{}}\PYG{p}{,} \PYG{l+s}{\PYGZsq{}}\PYG{l+s}{bolton}\PYG{l+s}{\PYGZsq{}}\PYG{p}{)}
\PYG{g+gp}{\PYGZgt{}\PYGZgt{}\PYGZgt{} }\PYG{n}{atmos}\PYG{o}{.}\PYG{n}{calculate}\PYG{p}{(}\PYG{l+s}{\PYGZsq{}}\PYG{l+s}{p}\PYG{l+s}{\PYGZsq{}}\PYG{p}{,} \PYG{n}{Tv}\PYG{o}{=}\PYG{l+m+mf}{273.}\PYG{p}{,} \PYG{n}{rho}\PYG{o}{=}\PYG{l+m+mf}{1.27}\PYG{p}{,} \PYG{n}{assumptions}\PYG{o}{=}\PYG{n}{assumptions}\PYG{p}{)}
\PYG{g+go}{99519.638400000011}
\end{Verbatim}


\section{Specifying quantities with a dictionary}
\label{calculate:specifying-quantities-with-a-dictionary}
If you are repeatedly calculating different quantities, you may want to use
a dictionary to more easily pass in quantities as keyword arguments. Adding
** to the beginning of a dictionary variable as an argument passes in
each of the (key, value) pairs in that dictionary as a separate keyword
argument. For example:

\begin{Verbatim}[commandchars=\\\{\}]
\PYG{g+gp}{\PYGZgt{}\PYGZgt{}\PYGZgt{} }\PYG{k+kn}{import} \PYG{n+nn}{atmos}
\PYG{g+gp}{\PYGZgt{}\PYGZgt{}\PYGZgt{} }\PYG{n}{data} \PYG{o}{=} \PYG{p}{\PYGZob{}}\PYG{l+s}{\PYGZsq{}}\PYG{l+s}{Tv}\PYG{l+s}{\PYGZsq{}}\PYG{p}{:} \PYG{l+m+mf}{273.}\PYG{p}{,} \PYG{l+s}{\PYGZsq{}}\PYG{l+s}{rho}\PYG{l+s}{\PYGZsq{}}\PYG{p}{:} \PYG{l+m+mf}{1.27}\PYG{p}{\PYGZcb{}}
\PYG{g+gp}{\PYGZgt{}\PYGZgt{}\PYGZgt{} }\PYG{n}{data}\PYG{p}{[}\PYG{l+s}{\PYGZsq{}}\PYG{l+s}{p}\PYG{l+s}{\PYGZsq{}}\PYG{p}{]} \PYG{o}{=} \PYG{n}{atmos}\PYG{o}{.}\PYG{n}{calculate}\PYG{p}{(}\PYG{l+s}{\PYGZsq{}}\PYG{l+s}{p}\PYG{l+s}{\PYGZsq{}}\PYG{p}{,} \PYG{o}{*}\PYG{o}{*}\PYG{n}{data}\PYG{p}{)}
\PYG{g+gp}{\PYGZgt{}\PYGZgt{}\PYGZgt{} }\PYG{n}{data}\PYG{p}{[}\PYG{l+s}{\PYGZsq{}}\PYG{l+s}{p}\PYG{l+s}{\PYGZsq{}}\PYG{p}{]}
\PYG{g+go}{99519.638400000011}
\end{Verbatim}


\section{Function reference}
\label{calculate:function-reference}
..autofunction:: atmos.calculate


\chapter{Using FluidSolver}
\label{solver::doc}\label{solver:using-fluidsolver}

\section{What does FluidSolver do?}
\label{solver:what-does-fluidsolver-do}
\code{atmos.FluidSolver} takes input variables (like
pressure, virtual temperature, water vapor mixing ratio, etc.) and information
about what assumptions you're willing to make (hydrostatic? low water vapor?
ignore virtual temperature correction? use an empirical formula for
equivalent potential temperature?), and from that calculates any desired
output variables that you request and can be calculated.

The main benefit of using \code{atmos.FluidSolver} instead of
\code{atmos.calculate()} is that the FluidSolver object has memory. It can keep
track of what assumptions you enabled, as well as what quantities you've given
it and it has calculated.


\section{What can it calculate?}
\label{solver:what-can-it-calculate}
Anything that can be calculated by equations in {\color{red}\bfseries{}:module:{}`atmos.equations{}`}.
If you find that the FluidSolver can't do a calculation you might expect it
to, check the equations it has available and make sure you're using the right
variables, or enabling the right assumptions. A common problem is using \emph{T}
instead of \emph{Tv} and expecting the ideal gas law to work.


\section{A simple example}
\label{solver:a-simple-example}
By default, a certain set of assumptions are used, such as that we are
considering an ideal gas, and so can use ideal gas law. This allows us to do
simple calculations that use the default assumptions. For example, to
calculate pressure from virtual temperature and density:

\begin{Verbatim}[commandchars=\\\{\}]
\PYG{g+gp}{\PYGZgt{}\PYGZgt{}\PYGZgt{} }\PYG{k+kn}{import} \PYG{n+nn}{atmos}
\PYG{g+gp}{\PYGZgt{}\PYGZgt{}\PYGZgt{} }\PYG{n}{solver} \PYG{o}{=} \PYG{n}{atmos}\PYG{o}{.}\PYG{n}{FluidSolver}\PYG{p}{(}\PYG{n}{Tv}\PYG{o}{=}\PYG{l+m+mf}{273.}\PYG{p}{,} \PYG{n}{rho}\PYG{o}{=}\PYG{l+m+mf}{1.27}\PYG{p}{)}
\PYG{g+gp}{\PYGZgt{}\PYGZgt{}\PYGZgt{} }\PYG{n}{solver}\PYG{o}{.}\PYG{n}{calculate}\PYG{p}{(}\PYG{l+s}{\PYGZsq{}}\PYG{l+s}{p}\PYG{l+s}{\PYGZsq{}}\PYG{p}{)}
\PYG{g+go}{99519.638400000011}
\end{Verbatim}

Or to calculate relative humidity from water vapor mixing ratio and
saturation water vapor mixing ratio (which needs no assumptions):

\begin{Verbatim}[commandchars=\\\{\}]
\PYG{g+gp}{\PYGZgt{}\PYGZgt{}\PYGZgt{} }\PYG{k+kn}{import} \PYG{n+nn}{atmos}
\PYG{g+gp}{\PYGZgt{}\PYGZgt{}\PYGZgt{} }\PYG{n}{solver} \PYG{o}{=} \PYG{n}{atmos}\PYG{o}{.}\PYG{n}{FluidSolver}\PYG{p}{(}\PYG{n}{rv}\PYG{o}{=}\PYG{l+m+mf}{0.001}\PYG{p}{,} \PYG{n}{rvs}\PYG{o}{=}\PYG{l+m+mf}{0.002}\PYG{p}{)}
\PYG{g+gp}{\PYGZgt{}\PYGZgt{}\PYGZgt{} }\PYG{n}{solver}\PYG{o}{.}\PYG{n}{calculate}\PYG{p}{(}\PYG{l+s}{\PYGZsq{}}\PYG{l+s}{RH}\PYG{l+s}{\PYGZsq{}}\PYG{p}{)}
\PYG{g+go}{50.0}
\end{Verbatim}

For a full list of default assumptions, see \code{atmos.FluidSolver}.


\section{Viewing equation functions used}
\label{solver:viewing-equation-functions-used}
Calculating pressure from virtual temperature and density, also returning a
list of functions used:

\begin{Verbatim}[commandchars=\\\{\}]
\PYG{g+gp}{\PYGZgt{}\PYGZgt{}\PYGZgt{} }\PYG{k+kn}{import} \PYG{n+nn}{atmos}
\PYG{g+gp}{\PYGZgt{}\PYGZgt{}\PYGZgt{} }\PYG{n}{solver} \PYG{o}{=} \PYG{n}{atmos}\PYG{o}{.}\PYG{n}{FluidSolver}\PYG{p}{(}\PYG{n}{Tv}\PYG{o}{=}\PYG{l+m+mf}{273.}\PYG{p}{,} \PYG{n}{rho}\PYG{o}{=}\PYG{l+m+mf}{1.27}\PYG{p}{,} \PYG{n}{debug}\PYG{o}{=}\PYG{n+nb+bp}{True}\PYG{p}{)}
\PYG{g+gp}{\PYGZgt{}\PYGZgt{}\PYGZgt{} }\PYG{n}{p}\PYG{p}{,} \PYG{n}{funcs} \PYG{o}{=} \PYG{n}{solver}\PYG{o}{.}\PYG{n}{calculate}\PYG{p}{(}\PYG{l+s}{\PYGZsq{}}\PYG{l+s}{p}\PYG{l+s}{\PYGZsq{}}\PYG{p}{)}
\PYG{g+gp}{\PYGZgt{}\PYGZgt{}\PYGZgt{} }\PYG{n}{funcs}
\PYG{g+go}{(\PYGZlt{}function atmos.equations.p\PYGZus{}from\PYGZus{}rho\PYGZus{}Tv\PYGZus{}ideal\PYGZus{}gas\PYGZgt{},)}
\end{Verbatim}


\section{Adding and removing assumptions}
\label{solver:adding-and-removing-assumptions}
If you want to use assumptions that are not enabled by default (such as
ignoring the virtual temperature correction), you can use the add\_assumptions
keyword argument, which takes a tuple of strings specifying assumptions.
The exact string to enter for each assumption is detailed in
\code{atmos.FluidSolver}. For example, to calculate T instead of Tv,
neglecting the virtual temperature correction:

\begin{Verbatim}[commandchars=\\\{\}]
\PYG{g+gp}{\PYGZgt{}\PYGZgt{}\PYGZgt{} }\PYG{k+kn}{import} \PYG{n+nn}{atmos}
\PYG{g+gp}{\PYGZgt{}\PYGZgt{}\PYGZgt{} }\PYG{n}{solver} \PYG{o}{=} \PYG{n}{atmos}\PYG{o}{.}\PYG{n}{FluidSolver}\PYG{p}{(}\PYG{n}{T}\PYG{o}{=}\PYG{l+m+mf}{273.}\PYG{p}{,} \PYG{n}{rho}\PYG{o}{=}\PYG{l+m+mf}{1.27}\PYG{p}{,}
\end{Verbatim}
\begin{description}
\item[{add\_assumptions=(`Tv equals T',))}] \leavevmode
\begin{Verbatim}[commandchars=\\\{\}]
\PYG{g+gp}{\PYGZgt{}\PYGZgt{}\PYGZgt{} }\PYG{n}{solver}\PYG{o}{.}\PYG{n}{calculate}\PYG{p}{(}\PYG{l+s}{\PYGZsq{}}\PYG{l+s}{p}\PYG{l+s}{\PYGZsq{}}\PYG{p}{)}
\PYG{g+go}{99519.638400000011}
\end{Verbatim}

\end{description}


\section{Overriding assumptions}
\label{solver:overriding-assumptions}
If you want to ignore the default assumptions entirely, you could specify
your own assumptions:

\begin{Verbatim}[commandchars=\\\{\}]
\PYG{g+gp}{\PYGZgt{}\PYGZgt{}\PYGZgt{} }\PYG{k+kn}{import} \PYG{n+nn}{atmos}
\PYG{g+gp}{\PYGZgt{}\PYGZgt{}\PYGZgt{} }\PYG{n}{solver} \PYG{o}{=} \PYG{n}{atmos}\PYG{o}{.}\PYG{n}{FluidSolver}\PYG{p}{(}\PYG{n}{Tv}\PYG{o}{=}\PYG{l+m+mf}{273.}\PYG{p}{,} \PYG{n}{rho}\PYG{o}{=}\PYG{l+m+mf}{1.27}\PYG{p}{,}
\end{Verbatim}
\begin{description}
\item[{assumptions=(`ideal gas', `bolton'))}] \leavevmode
\begin{Verbatim}[commandchars=\\\{\}]
\PYG{g+gp}{\PYGZgt{}\PYGZgt{}\PYGZgt{} }\PYG{n}{solver}\PYG{o}{.}\PYG{n}{calculate}\PYG{p}{(}\PYG{l+s}{\PYGZsq{}}\PYG{l+s}{p}\PYG{l+s}{\PYGZsq{}}\PYG{p}{)}
\PYG{g+go}{99519.638400000011}
\end{Verbatim}

\end{description}


\section{Class reference}
\label{solver:class-reference}
..autoclass:: atmos.Fluid Solver


\chapter{Subclassing BaseSolver and FluidSolver}
\label{subclassing::doc}\label{subclassing:subclassing-basesolver-and-fluidsolver}

\chapter{atmos package reference}
\label{atmos:atmos-package-reference}\label{atmos::doc}

\section{Module contents}
\label{atmos:module-atmos}\label{atmos:module-contents}\index{atmos (module)}

\subsection{atmos}
\label{atmos:atmos}

\subsubsection{An atmospheric sciences utility library}
\label{atmos:an-atmospheric-sciences-utility-library}
\textbf{atmos} is a library of Python programming utilities for the atmospheric
sciences. It is in ongoing development.

Information on how to use the module can be found predominantly by using the
built-in help() function in Python. Many docstrings are automatically
generated by the module and so information may appear to be missing in the
source code. HTML documentation will be available at a later date.

This module is currently alpha. The API of components at the base module
level should stay backwards-compatible, but sub-modules are subject to change.
In particular, features in the util module are likely to be changed or removed
entirely.


\section{Submodules}
\label{atmos:submodules}

\section{atmos.constants module}
\label{atmos:atmos-constants-module}\label{atmos:module-atmos.constants}\index{atmos.constants (module)}
constants.py: Scientific constants.


\section{atmos.decorators module}
\label{atmos:module-atmos.decorators}\label{atmos:atmos-decorators-module}\index{atmos.decorators (module)}
decorators.py: Function decorators used by the rest of this module.
\index{assumes() (in module atmos.decorators)}

\begin{fulllineitems}
\phantomsection\label{atmos:atmos.decorators.assumes}\pysiglinewithargsret{\code{atmos.decorators.}\bfcode{assumes}}{\emph{*args}}{}
Stores a function's assumptions as an attribute.

\end{fulllineitems}

\index{equation\_docstring() (in module atmos.decorators)}

\begin{fulllineitems}
\phantomsection\label{atmos:atmos.decorators.equation_docstring}\pysiglinewithargsret{\code{atmos.decorators.}\bfcode{equation\_docstring}}{\emph{quantity\_dict}, \emph{assumption\_dict}, \emph{equation=None}, \emph{references=None}, \emph{notes=None}}{}
Creates a decorator that adds a docstring to an equation function.
\begin{quote}\begin{description}
\item[{Parameters}] \leavevmode\begin{itemize}
\item {} 
\textbf{quantity\_dict} (\emph{dict}) -- A dictionary describing the quantities used in the equations. Its keys
should be abbreviations for the quantities, and its values should be a
dictionary of the form \{`name': string, `units': string\}.

\item {} 
\textbf{assumption\_dict} (\emph{dict}) -- A dictionary describing the assumptions used by the equations. Its keys
should be short forms of the assumptions, and its values should be long
forms of the assumptions, as you would insert into the sentence
`Calculates (quantity) assuming (assumption 1), (assumption 2), and
(assumption 3).'

\item {} 
\textbf{equation} (\emph{string, optional}) -- A string describing the equation the function uses. Should be wrapped
to be no more than 80 characters in length.

\item {} 
\textbf{references} (\emph{string, optional}) -- A string providing references for the function. Should be wrapped to be
no more than 80 characters in length.

\end{itemize}

\item[{Raises}] \leavevmode
\code{ValueError} --
If the function name does not follow (varname)\_from\_(any text here), or
if an argument of the function or the varname (as above) is not present
in quantity\_dict, or if an assumption in func.assumptions is not present
in the assumption\_dict.

\end{description}\end{quote}

\end{fulllineitems}

\index{overridden\_by\_assumptions() (in module atmos.decorators)}

\begin{fulllineitems}
\phantomsection\label{atmos:atmos.decorators.overridden_by_assumptions}\pysiglinewithargsret{\code{atmos.decorators.}\bfcode{overridden\_by\_assumptions}}{\emph{*args}}{}
Stores what assumptions a function is overridden by as an attribute.

\end{fulllineitems}



\section{atmos.equations module}
\label{atmos:module-atmos.equations}\label{atmos:atmos-equations-module}\index{atmos.equations (module)}
equations.py: Fluid dynamics equations for atmospheric sciences.
\index{AH\_from\_qv\_rho() (in module atmos.equations)}

\begin{fulllineitems}
\phantomsection\label{atmos:atmos.equations.AH_from_qv_rho}\pysiglinewithargsret{\code{atmos.equations.}\bfcode{AH\_from\_qv\_rho}}{\emph{qv}, \emph{rho}}{}
Calculates absolute humidity (kg/m\textasciicircum{}3).

\(AH = q_v \rho\)
\begin{quote}\begin{description}
\item[{Parameters}] \leavevmode\begin{itemize}
\item {} 
\textbf{qv} (\emph{float or ndarray}) -- Data for specific humidity (kg/kg).

\item {} 
\textbf{rho} (\emph{float or ndarray}) -- Data for density (kg/m\textasciicircum{}3).

\end{itemize}

\item[{Returns}] \leavevmode
\textbf{AH} --
Data for absolute humidity (kg/m\textasciicircum{}3).

\item[{Return type}] \leavevmode
float or ndarray

\end{description}\end{quote}

\end{fulllineitems}

\index{DSE\_from\_T\_Phi() (in module atmos.equations)}

\begin{fulllineitems}
\phantomsection\label{atmos:atmos.equations.DSE_from_T_Phi}\pysiglinewithargsret{\code{atmos.equations.}\bfcode{DSE\_from\_T\_Phi}}{\emph{T}, \emph{Phi}}{}
Calculates dry static energy (J).

\(DSE = C_{pd} T + \Phi\)
\begin{quote}\begin{description}
\item[{Parameters}] \leavevmode\begin{itemize}
\item {} 
\textbf{T} (\emph{float or ndarray}) -- Data for temperature (K).

\item {} 
\textbf{Phi} (\emph{float or ndarray}) -- Data for geopotential (m\textasciicircum{}2/s\textasciicircum{}2).

\end{itemize}

\item[{Returns}] \leavevmode
\textbf{DSE} --
Data for dry static energy (J).

\item[{Return type}] \leavevmode
float or ndarray

\end{description}\end{quote}

\end{fulllineitems}

\index{DSE\_from\_T\_z() (in module atmos.equations)}

\begin{fulllineitems}
\phantomsection\label{atmos:atmos.equations.DSE_from_T_z}\pysiglinewithargsret{\code{atmos.equations.}\bfcode{DSE\_from\_T\_z}}{\emph{T}, \emph{z}}{}
Calculates dry static energy (J) assuming g is constant.

\(DSE = C_{pd} T + g_0 z\)
\begin{quote}\begin{description}
\item[{Parameters}] \leavevmode\begin{itemize}
\item {} 
\textbf{T} (\emph{float or ndarray}) -- Data for temperature (K).

\item {} 
\textbf{z} (\emph{float or ndarray}) -- Data for height (m).

\end{itemize}

\item[{Returns}] \leavevmode
\textbf{DSE} --
Data for dry static energy (J).

\item[{Return type}] \leavevmode
float or ndarray

\end{description}\end{quote}

\end{fulllineitems}

\index{Gammam\_from\_rvs\_T() (in module atmos.equations)}

\begin{fulllineitems}
\phantomsection\label{atmos:atmos.equations.Gammam_from_rvs_T}\pysiglinewithargsret{\code{atmos.equations.}\bfcode{Gammam\_from\_rvs\_T}}{\emph{rvs}, \emph{T}}{}
Calculates moist adiabatic lapse rate (K/m) assuming g is constant and latent
heat of vaporization of water is constant.

\(Gammam = g_0 \frac{1+\frac{L_{v0}*r_{vs}}{R_d
T}}{C_{pd}+\frac{L_{v0}^2*r_{vs}}{R_v T^2}}\)
\begin{quote}\begin{description}
\item[{Parameters}] \leavevmode\begin{itemize}
\item {} 
\textbf{rvs} (\emph{float or ndarray}) -- Data for saturation water vapor mixing ratio (kg/kg).

\item {} 
\textbf{T} (\emph{float or ndarray}) -- Data for temperature (K).

\end{itemize}

\item[{Returns}] \leavevmode
\textbf{Gammam} --
Data for moist adiabatic lapse rate (K/m).

\item[{Return type}] \leavevmode
float or ndarray

\end{description}\end{quote}
\paragraph{Notes}

\textbf{References}
\begin{description}
\item[{American Meteorological Society Glossary of Meteorology}] \leavevmode
\href{http://glossary.ametsoc.org/wiki/Saturation-adiabatic\_lapse\_rate}{http://glossary.ametsoc.org/wiki/Saturation-adiabatic\_lapse\_rate}
Retrieved March 25, 2015

\end{description}

\end{fulllineitems}

\index{MSE\_from\_DSE\_qv() (in module atmos.equations)}

\begin{fulllineitems}
\phantomsection\label{atmos:atmos.equations.MSE_from_DSE_qv}\pysiglinewithargsret{\code{atmos.equations.}\bfcode{MSE\_from\_DSE\_qv}}{\emph{DSE}, \emph{qv}}{}
Calculates moist static energy (J) assuming latent heat of vaporization of water
is constant.

\(MSE = DSE + L_{v0} q_v\)
\begin{quote}\begin{description}
\item[{Parameters}] \leavevmode\begin{itemize}
\item {} 
\textbf{DSE} (\emph{float or ndarray}) -- Data for dry static energy (J).

\item {} 
\textbf{qv} (\emph{float or ndarray}) -- Data for specific humidity (kg/kg).

\end{itemize}

\item[{Returns}] \leavevmode
\textbf{MSE} --
Data for moist static energy (J).

\item[{Return type}] \leavevmode
float or ndarray

\end{description}\end{quote}

\end{fulllineitems}

\index{Phi\_from\_z() (in module atmos.equations)}

\begin{fulllineitems}
\phantomsection\label{atmos:atmos.equations.Phi_from_z}\pysiglinewithargsret{\code{atmos.equations.}\bfcode{Phi\_from\_z}}{\emph{z}}{}
Calculates geopotential (m\textasciicircum{}2/s\textasciicircum{}2) assuming g is constant.

\(Phi = g_0 z\)
\begin{quote}\begin{description}
\item[{Parameters}] \leavevmode
\textbf{z} (\emph{float or ndarray}) -- Data for height (m).

\item[{Returns}] \leavevmode
\textbf{Phi} --
Data for geopotential (m\textasciicircum{}2/s\textasciicircum{}2).

\item[{Return type}] \leavevmode
float or ndarray

\end{description}\end{quote}

\end{fulllineitems}

\index{RH\_from\_qv\_qvs\_lwv() (in module atmos.equations)}

\begin{fulllineitems}
\phantomsection\label{atmos:atmos.equations.RH_from_qv_qvs_lwv}\pysiglinewithargsret{\code{atmos.equations.}\bfcode{RH\_from\_qv\_qvs\_lwv}}{\emph{qv}, \emph{qvs}}{}
Calculates relative humidity (\%) assuming terms that are second-order in
moisture quantities can be neglected (eg. qv == rv).

\(RH = \frac{q_{v}}{q_{vs}} \times 100\)
\begin{quote}\begin{description}
\item[{Parameters}] \leavevmode\begin{itemize}
\item {} 
\textbf{qv} (\emph{float or ndarray}) -- Data for specific humidity (kg/kg).

\item {} 
\textbf{qvs} (\emph{float or ndarray}) -- Data for saturation specific humidity (kg/kg).

\end{itemize}

\item[{Returns}] \leavevmode
\textbf{RH} --
Data for relative humidity (\%).

\item[{Return type}] \leavevmode
float or ndarray

\end{description}\end{quote}

\end{fulllineitems}

\index{RH\_from\_rv\_rvs() (in module atmos.equations)}

\begin{fulllineitems}
\phantomsection\label{atmos:atmos.equations.RH_from_rv_rvs}\pysiglinewithargsret{\code{atmos.equations.}\bfcode{RH\_from\_rv\_rvs}}{\emph{rv}, \emph{rvs}}{}
Calculates relative humidity (\%).

\(RH = \frac{r_v}{r_{vs}} \times 100\)
\begin{quote}\begin{description}
\item[{Parameters}] \leavevmode\begin{itemize}
\item {} 
\textbf{rv} (\emph{float or ndarray}) -- Data for water vapor mixing ratio (kg/kg).

\item {} 
\textbf{rvs} (\emph{float or ndarray}) -- Data for saturation water vapor mixing ratio (kg/kg).

\end{itemize}

\item[{Returns}] \leavevmode
\textbf{RH} --
Data for relative humidity (\%).

\item[{Return type}] \leavevmode
float or ndarray

\end{description}\end{quote}

\end{fulllineitems}

\index{T\_from\_Tv\_assuming\_Tv\_equals\_T() (in module atmos.equations)}

\begin{fulllineitems}
\phantomsection\label{atmos:atmos.equations.T_from_Tv_assuming_Tv_equals_T}\pysiglinewithargsret{\code{atmos.equations.}\bfcode{T\_from\_Tv\_assuming\_Tv\_equals\_T}}{\emph{Tv}}{}
Calculates temperature (K) assuming the virtual temperature correction can be
neglected.

\(T = T_v\)
\begin{quote}\begin{description}
\item[{Parameters}] \leavevmode
\textbf{Tv} (\emph{float or ndarray}) -- Data for virtual temperature (K).

\item[{Returns}] \leavevmode
\textbf{T} --
Data for temperature (K).

\item[{Return type}] \leavevmode
float or ndarray

\end{description}\end{quote}
\paragraph{Notes}

This function exists to allow using temperature as virtual temperature.

\end{fulllineitems}

\index{T\_from\_Tv\_qv() (in module atmos.equations)}

\begin{fulllineitems}
\phantomsection\label{atmos:atmos.equations.T_from_Tv_qv}\pysiglinewithargsret{\code{atmos.equations.}\bfcode{T\_from\_Tv\_qv}}{\emph{Tv}, \emph{qv}}{}
Calculates temperature (K) assuming liquid water can be neglected and ice can be
neglected.

\(T = \frac{T_v}{1 + 0.608 q_v}\)
\begin{quote}\begin{description}
\item[{Parameters}] \leavevmode\begin{itemize}
\item {} 
\textbf{Tv} (\emph{float or ndarray}) -- Data for virtual temperature (K).

\item {} 
\textbf{qv} (\emph{float or ndarray}) -- Data for specific humidity (kg/kg).

\end{itemize}

\item[{Returns}] \leavevmode
\textbf{T} --
Data for temperature (K).

\item[{Return type}] \leavevmode
float or ndarray

\end{description}\end{quote}

\end{fulllineitems}

\index{T\_from\_es\_Bolton() (in module atmos.equations)}

\begin{fulllineitems}
\phantomsection\label{atmos:atmos.equations.T_from_es_Bolton}\pysiglinewithargsret{\code{atmos.equations.}\bfcode{T\_from\_es\_Bolton}}{\emph{es}}{}
Calculates temperature (K) assuming the assumptions in Bolton (1980) hold.

\(T = \frac{29.65 log(es)-4880.16}{log(es)-19.48}\)
\begin{quote}\begin{description}
\item[{Parameters}] \leavevmode
\textbf{es} (\emph{float or ndarray}) -- Data for saturation water vapor partial pressure (Pa).

\item[{Returns}] \leavevmode
\textbf{T} --
Data for temperature (K).

\item[{Return type}] \leavevmode
float or ndarray

\end{description}\end{quote}
\paragraph{Notes}

Fits Wexler's formula to an accuracy of 0.1\% for temperatures between
-35C and 35C.

\textbf{References}
\begin{description}
\item[{Bolton, D. 1980: The Computation of Equivalent Potential Temperature.}] \leavevmode
Mon. Wea. Rev., 108, 1046–1053.
doi: \href{http://dx.doi.org/10.1175/1520-0493(1980}{http://dx.doi.org/10.1175/1520-0493(1980})108\textless{}1046:TCOEPT\textgreater{}2.0.CO;2

\item[{Wexler, A. (1976): Vapor pressure formulation for water in range 0 to}] \leavevmode
100 C. A revision. J. Res. Natl. Bur. Stand. A, 80, 775-785.

\end{description}

\end{fulllineitems}

\index{T\_from\_p\_theta() (in module atmos.equations)}

\begin{fulllineitems}
\phantomsection\label{atmos:atmos.equations.T_from_p_theta}\pysiglinewithargsret{\code{atmos.equations.}\bfcode{T\_from\_p\_theta}}{\emph{p}, \emph{T}}{}
Calculates temperature (K) assuming Cp is constant and equal to Cp for dry air
at 0C.

\(T = \theta (\frac{10^5}{p})^{-\frac{R_d}{C_{pd}}}\)
\begin{quote}\begin{description}
\item[{Parameters}] \leavevmode\begin{itemize}
\item {} 
\textbf{p} (\emph{float or ndarray}) -- Data for pressure (Pa).

\item {} 
\textbf{T} (\emph{float or ndarray}) -- Data for temperature (K).

\end{itemize}

\item[{Returns}] \leavevmode
\textbf{T} --
Data for temperature (K).

\item[{Return type}] \leavevmode
float or ndarray

\end{description}\end{quote}

\end{fulllineitems}

\index{Tlcl\_from\_T\_RH() (in module atmos.equations)}

\begin{fulllineitems}
\phantomsection\label{atmos:atmos.equations.Tlcl_from_T_RH}\pysiglinewithargsret{\code{atmos.equations.}\bfcode{Tlcl\_from\_T\_RH}}{\emph{T}, \emph{RH}}{}
Calculates temperature at lifting condensation level (K) assuming the
assumptions in Bolton (1980) hold.

\(T_{lcl} = ((\frac{1}{T-55}-(\frac{log(\frac{RH}{100})}{2840}))^{-1} + 55\)
\begin{quote}\begin{description}
\item[{Parameters}] \leavevmode\begin{itemize}
\item {} 
\textbf{T} (\emph{float or ndarray}) -- Data for temperature (K).

\item {} 
\textbf{RH} (\emph{float or ndarray}) -- Data for relative humidity (\%).

\end{itemize}

\item[{Returns}] \leavevmode
\textbf{Tlcl} --
Data for temperature at lifting condensation level (K).

\item[{Return type}] \leavevmode
float or ndarray

\end{description}\end{quote}
\paragraph{Notes}

Uses Bolton (1980) equation 22.

\textbf{References}
\begin{description}
\item[{Bolton, D. 1980: The Computation of Equivalent Potential Temperature.}] \leavevmode
Mon. Wea. Rev., 108, 1046–1053.
doi: \href{http://dx.doi.org/10.1175/1520-0493(1980}{http://dx.doi.org/10.1175/1520-0493(1980})108\textless{}1046:TCOEPT\textgreater{}2.0.CO;2

\end{description}

\end{fulllineitems}

\index{Tlcl\_from\_T\_Td() (in module atmos.equations)}

\begin{fulllineitems}
\phantomsection\label{atmos:atmos.equations.Tlcl_from_T_Td}\pysiglinewithargsret{\code{atmos.equations.}\bfcode{Tlcl\_from\_T\_Td}}{\emph{T}, \emph{Td}}{}
Calculates temperature at lifting condensation level (K) assuming the
assumptions in Bolton (1980) hold.

\(T_{lcl} = ((1./(Td-56.))-(log(T/Td)/800.))^{-1} + 56.\)
\begin{quote}\begin{description}
\item[{Parameters}] \leavevmode\begin{itemize}
\item {} 
\textbf{T} (\emph{float or ndarray}) -- Data for temperature (K).

\item {} 
\textbf{Td} (\emph{float or ndarray}) -- Data for dewpoint temperature (K).

\end{itemize}

\item[{Returns}] \leavevmode
\textbf{Tlcl} --
Data for temperature at lifting condensation level (K).

\item[{Return type}] \leavevmode
float or ndarray

\end{description}\end{quote}
\paragraph{Notes}

Uses Bolton (1980) equation 15.

\textbf{References}
\begin{description}
\item[{Bolton, D. 1980: The Computation of Equivalent Potential Temperature.}] \leavevmode
Mon. Wea. Rev., 108, 1046–1053.
doi: \href{http://dx.doi.org/10.1175/1520-0493(1980}{http://dx.doi.org/10.1175/1520-0493(1980})108\textless{}1046:TCOEPT\textgreater{}2.0.CO;2

\end{description}

\end{fulllineitems}

\index{Tlcl\_from\_T\_e() (in module atmos.equations)}

\begin{fulllineitems}
\phantomsection\label{atmos:atmos.equations.Tlcl_from_T_e}\pysiglinewithargsret{\code{atmos.equations.}\bfcode{Tlcl\_from\_T\_e}}{\emph{T}, \emph{e}}{}
Calculates temperature at lifting condensation level (K) assuming the
assumptions in Bolton (1980) hold.

\(T_{lcl} = \frac{2840}{3.5 log(T) - log(e) - 4.805} + 55\)
\begin{quote}\begin{description}
\item[{Parameters}] \leavevmode\begin{itemize}
\item {} 
\textbf{T} (\emph{float or ndarray}) -- Data for temperature (K).

\item {} 
\textbf{e} (\emph{float or ndarray}) -- Data for water vapor partial pressure (Pa).

\end{itemize}

\item[{Returns}] \leavevmode
\textbf{Tlcl} --
Data for temperature at lifting condensation level (K).

\item[{Return type}] \leavevmode
float or ndarray

\end{description}\end{quote}
\paragraph{Notes}

Uses Bolton(1980) equation 21.

\textbf{References}
\begin{description}
\item[{Bolton, D. 1980: The Computation of Equivalent Potential Temperature.}] \leavevmode
Mon. Wea. Rev., 108, 1046–1053.
doi: \href{http://dx.doi.org/10.1175/1520-0493(1980}{http://dx.doi.org/10.1175/1520-0493(1980})108\textless{}1046:TCOEPT\textgreater{}2.0.CO;2

\end{description}

\end{fulllineitems}

\index{Tv\_from\_T\_assuming\_Tv\_equals\_T() (in module atmos.equations)}

\begin{fulllineitems}
\phantomsection\label{atmos:atmos.equations.Tv_from_T_assuming_Tv_equals_T}\pysiglinewithargsret{\code{atmos.equations.}\bfcode{Tv\_from\_T\_assuming\_Tv\_equals\_T}}{\emph{T}}{}
Calculates virtual temperature (K) assuming the virtual temperature correction
can be neglected.

\(T_v = T\)
\begin{quote}\begin{description}
\item[{Parameters}] \leavevmode
\textbf{T} (\emph{float or ndarray}) -- Data for temperature (K).

\item[{Returns}] \leavevmode
\textbf{Tv} --
Data for virtual temperature (K).

\item[{Return type}] \leavevmode
float or ndarray

\end{description}\end{quote}
\paragraph{Notes}

This function exists to allow using temperature as virtual temperature.

\end{fulllineitems}

\index{Tv\_from\_T\_qv() (in module atmos.equations)}

\begin{fulllineitems}
\phantomsection\label{atmos:atmos.equations.Tv_from_T_qv}\pysiglinewithargsret{\code{atmos.equations.}\bfcode{Tv\_from\_T\_qv}}{\emph{T}, \emph{qv}}{}
Calculates virtual temperature (K) assuming liquid water can be neglected and
ice can be neglected.

\(T_v = T*(1+0.608 q_v)\)
\begin{quote}\begin{description}
\item[{Parameters}] \leavevmode\begin{itemize}
\item {} 
\textbf{T} (\emph{float or ndarray}) -- Data for temperature (K).

\item {} 
\textbf{qv} (\emph{float or ndarray}) -- Data for specific humidity (kg/kg).

\end{itemize}

\item[{Returns}] \leavevmode
\textbf{Tv} --
Data for virtual temperature (K).

\item[{Return type}] \leavevmode
float or ndarray

\end{description}\end{quote}

\end{fulllineitems}

\index{Tv\_from\_p\_rho\_ideal\_gas() (in module atmos.equations)}

\begin{fulllineitems}
\phantomsection\label{atmos:atmos.equations.Tv_from_p_rho_ideal_gas}\pysiglinewithargsret{\code{atmos.equations.}\bfcode{Tv\_from\_p\_rho\_ideal\_gas}}{\emph{p}, \emph{rho}}{}
Calculates virtual temperature (K) assuming the ideal gas law holds.

\(T_v = \frac{p}{\rho R_d}\)
\begin{quote}\begin{description}
\item[{Parameters}] \leavevmode\begin{itemize}
\item {} 
\textbf{p} (\emph{float or ndarray}) -- Data for pressure (Pa).

\item {} 
\textbf{rho} (\emph{float or ndarray}) -- Data for density (kg/m\textasciicircum{}3).

\end{itemize}

\item[{Returns}] \leavevmode
\textbf{Tv} --
Data for virtual temperature (K).

\item[{Return type}] \leavevmode
float or ndarray

\end{description}\end{quote}

\end{fulllineitems}

\index{Tw\_from\_T\_RH\_Stull() (in module atmos.equations)}

\begin{fulllineitems}
\phantomsection\label{atmos:atmos.equations.Tw_from_T_RH_Stull}\pysiglinewithargsret{\code{atmos.equations.}\bfcode{Tw\_from\_T\_RH\_Stull}}{\emph{T}, \emph{RH}}{}
Calculates wet bulb temperature (K).
\begin{quote}\begin{description}
\item[{Parameters}] \leavevmode\begin{itemize}
\item {} 
\textbf{T} (\emph{float or ndarray}) -- Data for temperature (K).

\item {} 
\textbf{RH} (\emph{float or ndarray}) -- Data for relative humidity (\%).

\end{itemize}

\item[{Returns}] \leavevmode
\textbf{Tw} --
Data for wet bulb temperature (K).

\item[{Return type}] \leavevmode
float or ndarray

\end{description}\end{quote}
\paragraph{Notes}

Uses the empirical inverse solution from Stull (2011).

\textbf{References}
\begin{description}
\item[{Stull, R. 2011: Wet-Bulb Temperature from Relative Humidity and Air}] \leavevmode
Temperature. J. Appl. Meteor. Climatol., 50, 2267–2269.
doi: \href{http://dx.doi.org/10.1175/JAMC-D-11-0143.1}{http://dx.doi.org/10.1175/JAMC-D-11-0143.1}

\end{description}

\end{fulllineitems}

\index{autodoc() (in module atmos.equations)}

\begin{fulllineitems}
\phantomsection\label{atmos:atmos.equations.autodoc}\pysiglinewithargsret{\code{atmos.equations.}\bfcode{autodoc}}{\emph{**kwargs}}{}
\end{fulllineitems}

\index{e\_from\_Td\_Bolton() (in module atmos.equations)}

\begin{fulllineitems}
\phantomsection\label{atmos:atmos.equations.e_from_Td_Bolton}\pysiglinewithargsret{\code{atmos.equations.}\bfcode{e\_from\_Td\_Bolton}}{\emph{Td}}{}
Calculates water vapor partial pressure (Pa) assuming the assumptions in Bolton
(1980) hold.

\(e = es(Td)\)
\begin{quote}\begin{description}
\item[{Parameters}] \leavevmode
\textbf{Td} (\emph{float or ndarray}) -- Data for dewpoint temperature (K).

\item[{Returns}] \leavevmode
\textbf{e} --
Data for water vapor partial pressure (Pa).

\item[{Return type}] \leavevmode
float or ndarray

\end{description}\end{quote}

\end{fulllineitems}

\index{e\_from\_Td\_Goff\_Gratch() (in module atmos.equations)}

\begin{fulllineitems}
\phantomsection\label{atmos:atmos.equations.e_from_Td_Goff_Gratch}\pysiglinewithargsret{\code{atmos.equations.}\bfcode{e\_from\_Td\_Goff\_Gratch}}{\emph{Td}}{}
Calculates water vapor partial pressure (Pa) assuming the Goff-Gratch equation
for es.

\(e = es(Td)\)
\begin{quote}\begin{description}
\item[{Parameters}] \leavevmode
\textbf{Td} (\emph{float or ndarray}) -- Data for dewpoint temperature (K).

\item[{Returns}] \leavevmode
\textbf{e} --
Data for water vapor partial pressure (Pa).

\item[{Return type}] \leavevmode
float or ndarray

\end{description}\end{quote}
\paragraph{Notes}

\textbf{References}
\begin{description}
\item[{Goff, J. A., and Gratch, S. 1946: Low-pressure properties of water}] \leavevmode
from −160 to 212 °F, in Transactions of the American Society of
Heating and Ventilating Engineers, pp 95–122, presented at the
52nd annual meeting of the American Society of Heating and
Ventilating Engineers, New York, 1946.

\end{description}

\end{fulllineitems}

\index{e\_from\_p\_es\_T\_Tw() (in module atmos.equations)}

\begin{fulllineitems}
\phantomsection\label{atmos:atmos.equations.e_from_p_es_T_Tw}\pysiglinewithargsret{\code{atmos.equations.}\bfcode{e\_from\_p\_es\_T\_Tw}}{\emph{p}, \emph{es}, \emph{T}, \emph{Tw}}{}
Calculates water vapor partial pressure (Pa) assuming the bulb is not frozen.

\(e = es - (6.60 \times 10^{-4}) (1 + 0.00115 (T_w-273.15) (T-T_w)) p\)
\begin{quote}\begin{description}
\item[{Parameters}] \leavevmode\begin{itemize}
\item {} 
\textbf{p} (\emph{float or ndarray}) -- Data for pressure (Pa).

\item {} 
\textbf{es} (\emph{float or ndarray}) -- Data for saturation water vapor partial pressure (Pa).

\item {} 
\textbf{T} (\emph{float or ndarray}) -- Data for temperature (K).

\item {} 
\textbf{Tw} (\emph{float or ndarray}) -- Data for wet bulb temperature (K).

\end{itemize}

\item[{Returns}] \leavevmode
\textbf{e} --
Data for water vapor partial pressure (Pa).

\item[{Return type}] \leavevmode
float or ndarray

\end{description}\end{quote}
\paragraph{Notes}

\textbf{References}
\begin{description}
\item[{Petty, G.W. 2008: A First Course in Atmospheric Thermodynamics. 1st Ed.}] \leavevmode
Sundog Publishing.

\end{description}

\end{fulllineitems}

\index{e\_from\_p\_es\_T\_Tw\_frozen\_bulb() (in module atmos.equations)}

\begin{fulllineitems}
\phantomsection\label{atmos:atmos.equations.e_from_p_es_T_Tw_frozen_bulb}\pysiglinewithargsret{\code{atmos.equations.}\bfcode{e\_from\_p\_es\_T\_Tw\_frozen\_bulb}}{\emph{p}, \emph{es}, \emph{T}, \emph{Tw}}{}
Calculates water vapor partial pressure (Pa) assuming the bulb is frozen.

\(e = es - (5.82 \times 10^{-4}) (1 + 0.00115 (T_w-273.15)  (T-T_w)) p\)
\begin{quote}\begin{description}
\item[{Parameters}] \leavevmode\begin{itemize}
\item {} 
\textbf{p} (\emph{float or ndarray}) -- Data for pressure (Pa).

\item {} 
\textbf{es} (\emph{float or ndarray}) -- Data for saturation water vapor partial pressure (Pa).

\item {} 
\textbf{T} (\emph{float or ndarray}) -- Data for temperature (K).

\item {} 
\textbf{Tw} (\emph{float or ndarray}) -- Data for wet bulb temperature (K).

\end{itemize}

\item[{Returns}] \leavevmode
\textbf{e} --
Data for water vapor partial pressure (Pa).

\item[{Return type}] \leavevmode
float or ndarray

\end{description}\end{quote}
\paragraph{Notes}

\textbf{References}
\begin{description}
\item[{Petty, G.W. 2008: A First Course in Atmospheric Thermodynamics. 1st Ed.}] \leavevmode
Sundog Publishing.

\end{description}

\end{fulllineitems}

\index{e\_from\_p\_qv() (in module atmos.equations)}

\begin{fulllineitems}
\phantomsection\label{atmos:atmos.equations.e_from_p_qv}\pysiglinewithargsret{\code{atmos.equations.}\bfcode{e\_from\_p\_qv}}{\emph{p}, \emph{qv}}{}
Calculates water vapor partial pressure (Pa).

\(e = p \frac{q_v}{0.622+q_v}\)
\begin{quote}\begin{description}
\item[{Parameters}] \leavevmode\begin{itemize}
\item {} 
\textbf{p} (\emph{float or ndarray}) -- Data for pressure (Pa).

\item {} 
\textbf{qv} (\emph{float or ndarray}) -- Data for specific humidity (kg/kg).

\end{itemize}

\item[{Returns}] \leavevmode
\textbf{e} --
Data for water vapor partial pressure (Pa).

\item[{Return type}] \leavevmode
float or ndarray

\end{description}\end{quote}

\end{fulllineitems}

\index{es\_from\_T\_Bolton() (in module atmos.equations)}

\begin{fulllineitems}
\phantomsection\label{atmos:atmos.equations.es_from_T_Bolton}\pysiglinewithargsret{\code{atmos.equations.}\bfcode{es\_from\_T\_Bolton}}{\emph{T}}{}
Calculates saturation water vapor partial pressure (Pa) assuming the assumptions
in Bolton (1980) hold.

\(es(T) = 611.2 exp(17.67  rac{T-273.15}{T-29.65})\)
\begin{quote}\begin{description}
\item[{Parameters}] \leavevmode
\textbf{T} (\emph{float or ndarray}) -- Data for temperature (K).

\item[{Returns}] \leavevmode
\textbf{es} --
Data for saturation water vapor partial pressure (Pa).

\item[{Return type}] \leavevmode
float or ndarray

\end{description}\end{quote}
\paragraph{Notes}

Fits Wexler's formula to an accuracy of 0.1\% for temperatures between
-35C and 35C.

\textbf{References}
\begin{description}
\item[{Bolton, D. 1980: The Computation of Equivalent Potential Temperature.}] \leavevmode
Mon. Wea. Rev., 108, 1046–1053.
doi: \href{http://dx.doi.org/10.1175/1520-0493(1980}{http://dx.doi.org/10.1175/1520-0493(1980})108\textless{}1046:TCOEPT\textgreater{}2.0.CO;2

\item[{Wexler, A. (1976): Vapor pressure formulation for water in range 0 to}] \leavevmode
100 C. A revision. J. Res. Natl. Bur. Stand. A, 80, 775-785.

\end{description}

\end{fulllineitems}

\index{es\_from\_T\_Goff\_Gratch() (in module atmos.equations)}

\begin{fulllineitems}
\phantomsection\label{atmos:atmos.equations.es_from_T_Goff_Gratch}\pysiglinewithargsret{\code{atmos.equations.}\bfcode{es\_from\_T\_Goff\_Gratch}}{\emph{T}}{}
Calculates saturation water vapor partial pressure (Pa) assuming the Goff-Gratch
equation for es.
\begin{quote}\begin{description}
\item[{Parameters}] \leavevmode
\textbf{T} (\emph{float or ndarray}) -- Data for temperature (K).

\item[{Returns}] \leavevmode
\textbf{es} --
Data for saturation water vapor partial pressure (Pa).

\item[{Return type}] \leavevmode
float or ndarray

\end{description}\end{quote}
\paragraph{Notes}

The original Goff–Gratch (1946) equation reads as follows:

\begin{DUlineblock}{0em}
\item[] Log10(es) = -7.90298 (Tst/T-1)
\item[]
\begin{DUlineblock}{\DUlineblockindent}
\item[] + 5.02808 Log10(Tst/T)
\item[] - 1.3816*10-7 (10\textasciicircum{}(11.344 (1-T/Tst)) - 1)
\item[] + 8.1328*10-3 (10\textasciicircum{}(-3.49149 (Tst/T-1)) - 1)
\item[] + Log10(es\_st)
\end{DUlineblock}
\end{DUlineblock}

where:
* Log10 refers to the logarithm in base 10
* es is the saturation water vapor pressure (hPa)
* T is the absolute air temperature in kelvins
* Tst is the steam-point (i.e. boiling point at 1 atm.) temperature (373.16K)
* es\_st is es at the steam-point pressure (1 atm = 1013.25 hPa)

This formula is accurate but computationally intensive. For most purposes,
a more approximate formula is appropriate.

\textbf{References}
\begin{description}
\item[{Goff, J. A., and Gratch, S. 1946: Low-pressure properties of water}] \leavevmode
from −160 to 212 °F, in Transactions of the American Society of
Heating and Ventilating Engineers, pp 95–122, presented at the
52nd annual meeting of the American Society of Heating and
Ventilating Engineers, New York, 1946.

\item[{Goff, J. A. (1957) Saturation pressure of water on the new Kelvin}] \leavevmode
temperature scale, Transactions of the American Society of Heating and
Ventilating Engineers, pp 347–354, presented at the semi-annual meeting of
the American Society of Heating and Ventilating Engineers, Murray Bay, Que.
Canada.

\item[{World Meteorological Organization (1988) General meteorological}] \leavevmode
standards and recommended practices, Appendix A, WMO Technical Regulations,
WMO-No. 49.

\item[{World Meteorological Organization (2000) General meteorological}] \leavevmode
standards and recommended practices, Appendix A, WMO Technical Regulations,
WMO-No. 49, corrigendum.

\item[{Murphy, D. M. and Koop, T. (2005): Review of the vapour pressures of}] \leavevmode
ice and supercooled water for atmospheric applications, Quarterly Journal
of the Royal Meteorological Society 131(608): 1539–1565.
doi:10.1256/qj.04.94

\end{description}

\end{fulllineitems}

\index{f\_from\_lat() (in module atmos.equations)}

\begin{fulllineitems}
\phantomsection\label{atmos:atmos.equations.f_from_lat}\pysiglinewithargsret{\code{atmos.equations.}\bfcode{f\_from\_lat}}{\emph{lat}}{}
Calculates Coriolis parameter (Hz).

\(f = 2 \Omega sin(\frac{\pi}{180.} lat)\)
\begin{quote}\begin{description}
\item[{Parameters}] \leavevmode
\textbf{lat} (\emph{float or ndarray}) -- Data for latitude (degrees N).

\item[{Returns}] \leavevmode
\textbf{f} --
Data for Coriolis parameter (Hz).

\item[{Return type}] \leavevmode
float or ndarray

\end{description}\end{quote}

\end{fulllineitems}

\index{omega\_from\_w\_rho\_hydrostatic() (in module atmos.equations)}

\begin{fulllineitems}
\phantomsection\label{atmos:atmos.equations.omega_from_w_rho_hydrostatic}\pysiglinewithargsret{\code{atmos.equations.}\bfcode{omega\_from\_w\_rho\_hydrostatic}}{\emph{w}, \emph{rho}}{}
Calculates vertical velocity expressed as tendency of pressure (Pa/s) assuming
hydrostatic balance and g is constant.

\(\omega = - \rho g_0 w\)
\begin{quote}\begin{description}
\item[{Parameters}] \leavevmode\begin{itemize}
\item {} 
\textbf{w} (\emph{float or ndarray}) -- Data for vertical velocity (m/s).

\item {} 
\textbf{rho} (\emph{float or ndarray}) -- Data for density (kg/m\textasciicircum{}3).

\end{itemize}

\item[{Returns}] \leavevmode
\textbf{omega} --
Data for vertical velocity expressed as tendency of pressure (Pa/s).

\item[{Return type}] \leavevmode
float or ndarray

\end{description}\end{quote}

\end{fulllineitems}

\index{p\_from\_rho\_Tv\_ideal\_gas() (in module atmos.equations)}

\begin{fulllineitems}
\phantomsection\label{atmos:atmos.equations.p_from_rho_Tv_ideal_gas}\pysiglinewithargsret{\code{atmos.equations.}\bfcode{p\_from\_rho\_Tv\_ideal\_gas}}{\emph{rho}, \emph{Tv}}{}
Calculates pressure (Pa) assuming the ideal gas law holds.

\(p = \rho R_d T_v\)
\begin{quote}\begin{description}
\item[{Parameters}] \leavevmode\begin{itemize}
\item {} 
\textbf{rho} (\emph{float or ndarray}) -- Data for density (kg/m\textasciicircum{}3).

\item {} 
\textbf{Tv} (\emph{float or ndarray}) -- Data for virtual temperature (K).

\end{itemize}

\item[{Returns}] \leavevmode
\textbf{p} --
Data for pressure (Pa).

\item[{Return type}] \leavevmode
float or ndarray

\end{description}\end{quote}

\end{fulllineitems}

\index{plcl\_from\_p\_T\_Tlcl() (in module atmos.equations)}

\begin{fulllineitems}
\phantomsection\label{atmos:atmos.equations.plcl_from_p_T_Tlcl}\pysiglinewithargsret{\code{atmos.equations.}\bfcode{plcl\_from\_p\_T\_Tlcl}}{\emph{p}, \emph{T}, \emph{Tlcl}}{}
Calculates pressure at lifting condensation level (Pa) assuming Cp is constant
and equal to Cp for dry air at 0C.

\(p_{lcl} = p (\frac{T_{lcl}}{T})^(\frac{C_{pd}}{R_d})\)
\begin{quote}\begin{description}
\item[{Parameters}] \leavevmode\begin{itemize}
\item {} 
\textbf{p} (\emph{float or ndarray}) -- Data for pressure (Pa).

\item {} 
\textbf{T} (\emph{float or ndarray}) -- Data for temperature (K).

\item {} 
\textbf{Tlcl} (\emph{float or ndarray}) -- Data for temperature at lifting condensation level (K).

\end{itemize}

\item[{Returns}] \leavevmode
\textbf{plcl} --
Data for pressure at lifting condensation level (Pa).

\item[{Return type}] \leavevmode
float or ndarray

\end{description}\end{quote}

\end{fulllineitems}

\index{qi\_from\_qt\_qv() (in module atmos.equations)}

\begin{fulllineitems}
\phantomsection\label{atmos:atmos.equations.qi_from_qt_qv}\pysiglinewithargsret{\code{atmos.equations.}\bfcode{qi\_from\_qt\_qv}}{\emph{qt}, \emph{qv}}{}
Calculates specific humidity with respect to ice (kg/kg) assuming liquid water
can be neglected.

\(q_i = q_t-q_v\)
\begin{quote}\begin{description}
\item[{Parameters}] \leavevmode\begin{itemize}
\item {} 
\textbf{qt} (\emph{float or ndarray}) -- Data for specific humidity with respect to total water (kg/kg).

\item {} 
\textbf{qv} (\emph{float or ndarray}) -- Data for specific humidity (kg/kg).

\end{itemize}

\item[{Returns}] \leavevmode
\textbf{qi} --
Data for specific humidity with respect to ice (kg/kg).

\item[{Return type}] \leavevmode
float or ndarray

\end{description}\end{quote}

\end{fulllineitems}

\index{qi\_from\_qt\_qv\_ql() (in module atmos.equations)}

\begin{fulllineitems}
\phantomsection\label{atmos:atmos.equations.qi_from_qt_qv_ql}\pysiglinewithargsret{\code{atmos.equations.}\bfcode{qi\_from\_qt\_qv\_ql}}{\emph{qt}, \emph{qv}, \emph{ql}}{}
Calculates specific humidity with respect to ice (kg/kg).

\(q_i = q_t-q_v-q_l\)
\begin{quote}\begin{description}
\item[{Parameters}] \leavevmode\begin{itemize}
\item {} 
\textbf{qt} (\emph{float or ndarray}) -- Data for specific humidity with respect to total water (kg/kg).

\item {} 
\textbf{qv} (\emph{float or ndarray}) -- Data for specific humidity (kg/kg).

\item {} 
\textbf{ql} (\emph{float or ndarray}) -- Data for specific humidity with respect to liquid water (kg/kg).

\end{itemize}

\item[{Returns}] \leavevmode
\textbf{qi} --
Data for specific humidity with respect to ice (kg/kg).

\item[{Return type}] \leavevmode
float or ndarray

\end{description}\end{quote}

\end{fulllineitems}

\index{ql\_from\_qt\_qv() (in module atmos.equations)}

\begin{fulllineitems}
\phantomsection\label{atmos:atmos.equations.ql_from_qt_qv}\pysiglinewithargsret{\code{atmos.equations.}\bfcode{ql\_from\_qt\_qv}}{\emph{qt}, \emph{qv}}{}
Calculates specific humidity with respect to liquid water (kg/kg) assuming ice
can be neglected.

\(q_l = q_t-q_v\)
\begin{quote}\begin{description}
\item[{Parameters}] \leavevmode\begin{itemize}
\item {} 
\textbf{qt} (\emph{float or ndarray}) -- Data for specific humidity with respect to total water (kg/kg).

\item {} 
\textbf{qv} (\emph{float or ndarray}) -- Data for specific humidity (kg/kg).

\end{itemize}

\item[{Returns}] \leavevmode
\textbf{ql} --
Data for specific humidity with respect to liquid water (kg/kg).

\item[{Return type}] \leavevmode
float or ndarray

\end{description}\end{quote}

\end{fulllineitems}

\index{ql\_from\_qt\_qv\_qi() (in module atmos.equations)}

\begin{fulllineitems}
\phantomsection\label{atmos:atmos.equations.ql_from_qt_qv_qi}\pysiglinewithargsret{\code{atmos.equations.}\bfcode{ql\_from\_qt\_qv\_qi}}{\emph{qt}, \emph{qv}, \emph{qi}}{}
Calculates specific humidity with respect to liquid water (kg/kg).

\(q_l = q_t-q_v-q_i\)
\begin{quote}\begin{description}
\item[{Parameters}] \leavevmode\begin{itemize}
\item {} 
\textbf{qt} (\emph{float or ndarray}) -- Data for specific humidity with respect to total water (kg/kg).

\item {} 
\textbf{qv} (\emph{float or ndarray}) -- Data for specific humidity (kg/kg).

\item {} 
\textbf{qi} (\emph{float or ndarray}) -- Data for specific humidity with respect to ice (kg/kg).

\end{itemize}

\item[{Returns}] \leavevmode
\textbf{ql} --
Data for specific humidity with respect to liquid water (kg/kg).

\item[{Return type}] \leavevmode
float or ndarray

\end{description}\end{quote}

\end{fulllineitems}

\index{qt\_from\_qi\_qv\_ql() (in module atmos.equations)}

\begin{fulllineitems}
\phantomsection\label{atmos:atmos.equations.qt_from_qi_qv_ql}\pysiglinewithargsret{\code{atmos.equations.}\bfcode{qt\_from\_qi\_qv\_ql}}{\emph{qi}, \emph{qv}, \emph{ql}}{}
Calculates specific humidity with respect to total water (kg/kg).

\(q_t = q_i+q_v+q_l\)
\begin{quote}\begin{description}
\item[{Parameters}] \leavevmode\begin{itemize}
\item {} 
\textbf{qi} (\emph{float or ndarray}) -- Data for specific humidity with respect to ice (kg/kg).

\item {} 
\textbf{qv} (\emph{float or ndarray}) -- Data for specific humidity (kg/kg).

\item {} 
\textbf{ql} (\emph{float or ndarray}) -- Data for specific humidity with respect to liquid water (kg/kg).

\end{itemize}

\item[{Returns}] \leavevmode
\textbf{qt} --
Data for specific humidity with respect to total water (kg/kg).

\item[{Return type}] \leavevmode
float or ndarray

\end{description}\end{quote}

\end{fulllineitems}

\index{qt\_from\_qv() (in module atmos.equations)}

\begin{fulllineitems}
\phantomsection\label{atmos:atmos.equations.qt_from_qv}\pysiglinewithargsret{\code{atmos.equations.}\bfcode{qt\_from\_qv}}{\emph{qv}}{}
Calculates specific humidity with respect to total water (kg/kg) assuming liquid
water can be neglected and ice can be neglected.

\(q_t = q_v\)
\begin{quote}\begin{description}
\item[{Parameters}] \leavevmode
\textbf{qv} (\emph{float or ndarray}) -- Data for specific humidity (kg/kg).

\item[{Returns}] \leavevmode
\textbf{qt} --
Data for specific humidity with respect to total water (kg/kg).

\item[{Return type}] \leavevmode
float or ndarray

\end{description}\end{quote}

\end{fulllineitems}

\index{qt\_from\_qv\_qi() (in module atmos.equations)}

\begin{fulllineitems}
\phantomsection\label{atmos:atmos.equations.qt_from_qv_qi}\pysiglinewithargsret{\code{atmos.equations.}\bfcode{qt\_from\_qv\_qi}}{\emph{qv}, \emph{qi}}{}
Calculates specific humidity with respect to total water (kg/kg) assuming liquid
water can be neglected.

\(q_t = q_v+q_l\)
\begin{quote}\begin{description}
\item[{Parameters}] \leavevmode\begin{itemize}
\item {} 
\textbf{qv} (\emph{float or ndarray}) -- Data for specific humidity (kg/kg).

\item {} 
\textbf{qi} (\emph{float or ndarray}) -- Data for specific humidity with respect to ice (kg/kg).

\end{itemize}

\item[{Returns}] \leavevmode
\textbf{qt} --
Data for specific humidity with respect to total water (kg/kg).

\item[{Return type}] \leavevmode
float or ndarray

\end{description}\end{quote}

\end{fulllineitems}

\index{qt\_from\_qv\_ql() (in module atmos.equations)}

\begin{fulllineitems}
\phantomsection\label{atmos:atmos.equations.qt_from_qv_ql}\pysiglinewithargsret{\code{atmos.equations.}\bfcode{qt\_from\_qv\_ql}}{\emph{qv}, \emph{ql}}{}
Calculates specific humidity with respect to total water (kg/kg) assuming ice
can be neglected.

\(q_t = q_v+q_l\)
\begin{quote}\begin{description}
\item[{Parameters}] \leavevmode\begin{itemize}
\item {} 
\textbf{qv} (\emph{float or ndarray}) -- Data for specific humidity (kg/kg).

\item {} 
\textbf{ql} (\emph{float or ndarray}) -- Data for specific humidity with respect to liquid water (kg/kg).

\end{itemize}

\item[{Returns}] \leavevmode
\textbf{qt} --
Data for specific humidity with respect to total water (kg/kg).

\item[{Return type}] \leavevmode
float or ndarray

\end{description}\end{quote}

\end{fulllineitems}

\index{qv\_from\_AH\_rho() (in module atmos.equations)}

\begin{fulllineitems}
\phantomsection\label{atmos:atmos.equations.qv_from_AH_rho}\pysiglinewithargsret{\code{atmos.equations.}\bfcode{qv\_from\_AH\_rho}}{\emph{AH}, \emph{rho}}{}
Calculates specific humidity (kg/kg).

\(q_v = \frac{AH}{\rho}\)
\begin{quote}\begin{description}
\item[{Parameters}] \leavevmode\begin{itemize}
\item {} 
\textbf{AH} (\emph{float or ndarray}) -- Data for absolute humidity (kg/m\textasciicircum{}3).

\item {} 
\textbf{rho} (\emph{float or ndarray}) -- Data for density (kg/m\textasciicircum{}3).

\end{itemize}

\item[{Returns}] \leavevmode
\textbf{qv} --
Data for specific humidity (kg/kg).

\item[{Return type}] \leavevmode
float or ndarray

\end{description}\end{quote}

\end{fulllineitems}

\index{qv\_from\_Tv\_T() (in module atmos.equations)}

\begin{fulllineitems}
\phantomsection\label{atmos:atmos.equations.qv_from_Tv_T}\pysiglinewithargsret{\code{atmos.equations.}\bfcode{qv\_from\_Tv\_T}}{\emph{Tv}, \emph{T}}{}
Calculates specific humidity (kg/kg) assuming liquid water can be neglected and
ice can be neglected.

\(qv = \frac{(\frac{Tv}{T} - 1)}{0.608}\)
\begin{quote}\begin{description}
\item[{Parameters}] \leavevmode\begin{itemize}
\item {} 
\textbf{Tv} (\emph{float or ndarray}) -- Data for virtual temperature (K).

\item {} 
\textbf{T} (\emph{float or ndarray}) -- Data for temperature (K).

\end{itemize}

\item[{Returns}] \leavevmode
\textbf{qv} --
Data for specific humidity (kg/kg).

\item[{Return type}] \leavevmode
float or ndarray

\end{description}\end{quote}

\end{fulllineitems}

\index{qv\_from\_p\_e() (in module atmos.equations)}

\begin{fulllineitems}
\phantomsection\label{atmos:atmos.equations.qv_from_p_e}\pysiglinewithargsret{\code{atmos.equations.}\bfcode{qv\_from\_p\_e}}{\emph{p}, \emph{e}}{}
Calculates specific humidity (kg/kg).

\(q_v = \frac{R_d}{R_v} \frac{e}{p-(1-\frac{R_d}{R_v}) e}\)
\begin{quote}\begin{description}
\item[{Parameters}] \leavevmode\begin{itemize}
\item {} 
\textbf{p} (\emph{float or ndarray}) -- Data for pressure (Pa).

\item {} 
\textbf{e} (\emph{float or ndarray}) -- Data for water vapor partial pressure (Pa).

\end{itemize}

\item[{Returns}] \leavevmode
\textbf{qv} --
Data for specific humidity (kg/kg).

\item[{Return type}] \leavevmode
float or ndarray

\end{description}\end{quote}

\end{fulllineitems}

\index{qv\_from\_p\_e\_lwv() (in module atmos.equations)}

\begin{fulllineitems}
\phantomsection\label{atmos:atmos.equations.qv_from_p_e_lwv}\pysiglinewithargsret{\code{atmos.equations.}\bfcode{qv\_from\_p\_e\_lwv}}{\emph{p}, \emph{e}}{}
Calculates specific humidity (kg/kg) assuming terms that are second-order in
moisture quantities can be neglected (eg. qv == rv).

\(qv = (\frac{R_d}{R_v}) \frac{e}{p}\)
\begin{quote}\begin{description}
\item[{Parameters}] \leavevmode\begin{itemize}
\item {} 
\textbf{p} (\emph{float or ndarray}) -- Data for pressure (Pa).

\item {} 
\textbf{e} (\emph{float or ndarray}) -- Data for water vapor partial pressure (Pa).

\end{itemize}

\item[{Returns}] \leavevmode
\textbf{qv} --
Data for specific humidity (kg/kg).

\item[{Return type}] \leavevmode
float or ndarray

\end{description}\end{quote}

\end{fulllineitems}

\index{qv\_from\_qt() (in module atmos.equations)}

\begin{fulllineitems}
\phantomsection\label{atmos:atmos.equations.qv_from_qt}\pysiglinewithargsret{\code{atmos.equations.}\bfcode{qv\_from\_qt}}{\emph{qt}}{}
Calculates specific humidity (kg/kg) assuming liquid water can be neglected and
ice can be neglected.

\(q_v = q_t\)
\begin{quote}\begin{description}
\item[{Parameters}] \leavevmode
\textbf{qt} (\emph{float or ndarray}) -- Data for specific humidity with respect to total water (kg/kg).

\item[{Returns}] \leavevmode
\textbf{qv} --
Data for specific humidity (kg/kg).

\item[{Return type}] \leavevmode
float or ndarray

\end{description}\end{quote}

\end{fulllineitems}

\index{qv\_from\_qt\_qi() (in module atmos.equations)}

\begin{fulllineitems}
\phantomsection\label{atmos:atmos.equations.qv_from_qt_qi}\pysiglinewithargsret{\code{atmos.equations.}\bfcode{qv\_from\_qt\_qi}}{\emph{qt}, \emph{qi}}{}
Calculates specific humidity (kg/kg) assuming liquid water can be neglected.

\(q_v = q_t - q_i\)
\begin{quote}\begin{description}
\item[{Parameters}] \leavevmode\begin{itemize}
\item {} 
\textbf{qt} (\emph{float or ndarray}) -- Data for specific humidity with respect to total water (kg/kg).

\item {} 
\textbf{qi} (\emph{float or ndarray}) -- Data for specific humidity with respect to ice (kg/kg).

\end{itemize}

\item[{Returns}] \leavevmode
\textbf{qv} --
Data for specific humidity (kg/kg).

\item[{Return type}] \leavevmode
float or ndarray

\end{description}\end{quote}

\end{fulllineitems}

\index{qv\_from\_qt\_ql() (in module atmos.equations)}

\begin{fulllineitems}
\phantomsection\label{atmos:atmos.equations.qv_from_qt_ql}\pysiglinewithargsret{\code{atmos.equations.}\bfcode{qv\_from\_qt\_ql}}{\emph{qt}, \emph{ql}}{}
Calculates specific humidity (kg/kg) assuming ice can be neglected.

\(q_v = q_t-q_l\)
\begin{quote}\begin{description}
\item[{Parameters}] \leavevmode\begin{itemize}
\item {} 
\textbf{qt} (\emph{float or ndarray}) -- Data for specific humidity with respect to total water (kg/kg).

\item {} 
\textbf{ql} (\emph{float or ndarray}) -- Data for specific humidity with respect to liquid water (kg/kg).

\end{itemize}

\item[{Returns}] \leavevmode
\textbf{qv} --
Data for specific humidity (kg/kg).

\item[{Return type}] \leavevmode
float or ndarray

\end{description}\end{quote}

\end{fulllineitems}

\index{qv\_from\_qt\_ql\_qi() (in module atmos.equations)}

\begin{fulllineitems}
\phantomsection\label{atmos:atmos.equations.qv_from_qt_ql_qi}\pysiglinewithargsret{\code{atmos.equations.}\bfcode{qv\_from\_qt\_ql\_qi}}{\emph{qt}, \emph{ql}, \emph{qi}}{}
Calculates specific humidity (kg/kg).

\(q_v = q_t-q_l-q_i\)
\begin{quote}\begin{description}
\item[{Parameters}] \leavevmode\begin{itemize}
\item {} 
\textbf{qt} (\emph{float or ndarray}) -- Data for specific humidity with respect to total water (kg/kg).

\item {} 
\textbf{ql} (\emph{float or ndarray}) -- Data for specific humidity with respect to liquid water (kg/kg).

\item {} 
\textbf{qi} (\emph{float or ndarray}) -- Data for specific humidity with respect to ice (kg/kg).

\end{itemize}

\item[{Returns}] \leavevmode
\textbf{qv} --
Data for specific humidity (kg/kg).

\item[{Return type}] \leavevmode
float or ndarray

\end{description}\end{quote}

\end{fulllineitems}

\index{qv\_from\_rv() (in module atmos.equations)}

\begin{fulllineitems}
\phantomsection\label{atmos:atmos.equations.qv_from_rv}\pysiglinewithargsret{\code{atmos.equations.}\bfcode{qv\_from\_rv}}{\emph{rv}}{}
Calculates specific humidity (kg/kg).

\(q_v = \frac{r_v}{1+r_v}\)
\begin{quote}\begin{description}
\item[{Parameters}] \leavevmode
\textbf{rv} (\emph{float or ndarray}) -- Data for water vapor mixing ratio (kg/kg).

\item[{Returns}] \leavevmode
\textbf{qv} --
Data for specific humidity (kg/kg).

\item[{Return type}] \leavevmode
float or ndarray

\end{description}\end{quote}

\end{fulllineitems}

\index{qv\_from\_rv\_lwv() (in module atmos.equations)}

\begin{fulllineitems}
\phantomsection\label{atmos:atmos.equations.qv_from_rv_lwv}\pysiglinewithargsret{\code{atmos.equations.}\bfcode{qv\_from\_rv\_lwv}}{\emph{rv}}{}
Calculates specific humidity (kg/kg) assuming terms that are second-order in
moisture quantities can be neglected (eg. qv == rv).

\(q_v = r_v\)
\begin{quote}\begin{description}
\item[{Parameters}] \leavevmode
\textbf{rv} (\emph{float or ndarray}) -- Data for water vapor mixing ratio (kg/kg).

\item[{Returns}] \leavevmode
\textbf{qv} --
Data for specific humidity (kg/kg).

\item[{Return type}] \leavevmode
float or ndarray

\end{description}\end{quote}

\end{fulllineitems}

\index{qvs\_from\_p\_es() (in module atmos.equations)}

\begin{fulllineitems}
\phantomsection\label{atmos:atmos.equations.qvs_from_p_es}\pysiglinewithargsret{\code{atmos.equations.}\bfcode{qvs\_from\_p\_es}}{\emph{p}, \emph{es}}{}
Calculates saturation specific humidity (kg/kg).

\(q_{vs} = qv\_from\_p\_e(p, e_s)\)
\begin{quote}\begin{description}
\item[{Parameters}] \leavevmode\begin{itemize}
\item {} 
\textbf{p} (\emph{float or ndarray}) -- Data for pressure (Pa).

\item {} 
\textbf{es} (\emph{float or ndarray}) -- Data for saturation water vapor partial pressure (Pa).

\end{itemize}

\item[{Returns}] \leavevmode
\textbf{qvs} --
Data for saturation specific humidity (kg/kg).

\item[{Return type}] \leavevmode
float or ndarray

\end{description}\end{quote}

\end{fulllineitems}

\index{qvs\_from\_p\_es\_lwv() (in module atmos.equations)}

\begin{fulllineitems}
\phantomsection\label{atmos:atmos.equations.qvs_from_p_es_lwv}\pysiglinewithargsret{\code{atmos.equations.}\bfcode{qvs\_from\_p\_es\_lwv}}{\emph{p}, \emph{es}}{}
Calculates saturation specific humidity (kg/kg) assuming terms that are second-
order in moisture quantities can be neglected (eg. qv == rv).

\(q_v = qv\_from\_p\_e\_lwv(p, e_s)\)
\begin{quote}\begin{description}
\item[{Parameters}] \leavevmode\begin{itemize}
\item {} 
\textbf{p} (\emph{float or ndarray}) -- Data for pressure (Pa).

\item {} 
\textbf{es} (\emph{float or ndarray}) -- Data for saturation water vapor partial pressure (Pa).

\end{itemize}

\item[{Returns}] \leavevmode
\textbf{qvs} --
Data for saturation specific humidity (kg/kg).

\item[{Return type}] \leavevmode
float or ndarray

\end{description}\end{quote}

\end{fulllineitems}

\index{qvs\_from\_rvs() (in module atmos.equations)}

\begin{fulllineitems}
\phantomsection\label{atmos:atmos.equations.qvs_from_rvs}\pysiglinewithargsret{\code{atmos.equations.}\bfcode{qvs\_from\_rvs}}{\emph{rvs}}{}
Calculates saturation specific humidity (kg/kg).

\(q_{vs} = \frac{r_{vs}}{1+r_{vs}}\)
\begin{quote}\begin{description}
\item[{Parameters}] \leavevmode
\textbf{rvs} (\emph{float or ndarray}) -- Data for saturation water vapor mixing ratio (kg/kg).

\item[{Returns}] \leavevmode
\textbf{qvs} --
Data for saturation specific humidity (kg/kg).

\item[{Return type}] \leavevmode
float or ndarray

\end{description}\end{quote}

\end{fulllineitems}

\index{qvs\_from\_rvs\_lwv() (in module atmos.equations)}

\begin{fulllineitems}
\phantomsection\label{atmos:atmos.equations.qvs_from_rvs_lwv}\pysiglinewithargsret{\code{atmos.equations.}\bfcode{qvs\_from\_rvs\_lwv}}{\emph{rvs}}{}
Calculates saturation specific humidity (kg/kg) assuming terms that are second-
order in moisture quantities can be neglected (eg. qv == rv).

\(q_v = r_v\)
\begin{quote}\begin{description}
\item[{Parameters}] \leavevmode
\textbf{rvs} (\emph{float or ndarray}) -- Data for saturation water vapor mixing ratio (kg/kg).

\item[{Returns}] \leavevmode
\textbf{qvs} --
Data for saturation specific humidity (kg/kg).

\item[{Return type}] \leavevmode
float or ndarray

\end{description}\end{quote}

\end{fulllineitems}

\index{rho\_from\_p\_Tv\_ideal\_gas() (in module atmos.equations)}

\begin{fulllineitems}
\phantomsection\label{atmos:atmos.equations.rho_from_p_Tv_ideal_gas}\pysiglinewithargsret{\code{atmos.equations.}\bfcode{rho\_from\_p\_Tv\_ideal\_gas}}{\emph{p}, \emph{Tv}}{}
Calculates density (kg/m\textasciicircum{}3) assuming the ideal gas law holds.

\(\rho = \frac{p}{R_d T_v}\)
\begin{quote}\begin{description}
\item[{Parameters}] \leavevmode\begin{itemize}
\item {} 
\textbf{p} (\emph{float or ndarray}) -- Data for pressure (Pa).

\item {} 
\textbf{Tv} (\emph{float or ndarray}) -- Data for virtual temperature (K).

\end{itemize}

\item[{Returns}] \leavevmode
\textbf{rho} --
Data for density (kg/m\textasciicircum{}3).

\item[{Return type}] \leavevmode
float or ndarray

\end{description}\end{quote}

\end{fulllineitems}

\index{rho\_from\_qv\_AH() (in module atmos.equations)}

\begin{fulllineitems}
\phantomsection\label{atmos:atmos.equations.rho_from_qv_AH}\pysiglinewithargsret{\code{atmos.equations.}\bfcode{rho\_from\_qv\_AH}}{\emph{qv}, \emph{AH}}{}
Calculates density (kg/m\textasciicircum{}3).

\(\rho = \frac{AH}{q_v}\)
\begin{quote}\begin{description}
\item[{Parameters}] \leavevmode\begin{itemize}
\item {} 
\textbf{qv} (\emph{float or ndarray}) -- Data for specific humidity (kg/kg).

\item {} 
\textbf{AH} (\emph{float or ndarray}) -- Data for absolute humidity (kg/m\textasciicircum{}3).

\end{itemize}

\item[{Returns}] \leavevmode
\textbf{rho} --
Data for density (kg/m\textasciicircum{}3).

\item[{Return type}] \leavevmode
float or ndarray

\end{description}\end{quote}

\end{fulllineitems}

\index{ri\_from\_rt\_rv() (in module atmos.equations)}

\begin{fulllineitems}
\phantomsection\label{atmos:atmos.equations.ri_from_rt_rv}\pysiglinewithargsret{\code{atmos.equations.}\bfcode{ri\_from\_rt\_rv}}{\emph{rt}, \emph{rv}}{}
Calculates ice mixing ratio (kg/kg) assuming liquid water can be neglected.

\(r_i = r_t-r_v\)
\begin{quote}\begin{description}
\item[{Parameters}] \leavevmode\begin{itemize}
\item {} 
\textbf{rt} (\emph{float or ndarray}) -- Data for total water mixing ratio (kg/kg).

\item {} 
\textbf{rv} (\emph{float or ndarray}) -- Data for water vapor mixing ratio (kg/kg).

\end{itemize}

\item[{Returns}] \leavevmode
\textbf{ri} --
Data for ice mixing ratio (kg/kg).

\item[{Return type}] \leavevmode
float or ndarray

\end{description}\end{quote}

\end{fulllineitems}

\index{ri\_from\_rt\_rv\_rl() (in module atmos.equations)}

\begin{fulllineitems}
\phantomsection\label{atmos:atmos.equations.ri_from_rt_rv_rl}\pysiglinewithargsret{\code{atmos.equations.}\bfcode{ri\_from\_rt\_rv\_rl}}{\emph{rt}, \emph{rv}, \emph{rl}}{}
Calculates ice mixing ratio (kg/kg).

\(r_i = r_t-r_v-r_l\)
\begin{quote}\begin{description}
\item[{Parameters}] \leavevmode\begin{itemize}
\item {} 
\textbf{rt} (\emph{float or ndarray}) -- Data for total water mixing ratio (kg/kg).

\item {} 
\textbf{rv} (\emph{float or ndarray}) -- Data for water vapor mixing ratio (kg/kg).

\item {} 
\textbf{rl} (\emph{float or ndarray}) -- Data for liquid water mixing ratio (kg/kg).

\end{itemize}

\item[{Returns}] \leavevmode
\textbf{ri} --
Data for ice mixing ratio (kg/kg).

\item[{Return type}] \leavevmode
float or ndarray

\end{description}\end{quote}

\end{fulllineitems}

\index{rl\_from\_rt\_rv() (in module atmos.equations)}

\begin{fulllineitems}
\phantomsection\label{atmos:atmos.equations.rl_from_rt_rv}\pysiglinewithargsret{\code{atmos.equations.}\bfcode{rl\_from\_rt\_rv}}{\emph{rt}, \emph{rv}}{}
Calculates liquid water mixing ratio (kg/kg) assuming ice can be neglected.

\(r_l = r_t-r_v\)
\begin{quote}\begin{description}
\item[{Parameters}] \leavevmode\begin{itemize}
\item {} 
\textbf{rt} (\emph{float or ndarray}) -- Data for total water mixing ratio (kg/kg).

\item {} 
\textbf{rv} (\emph{float or ndarray}) -- Data for water vapor mixing ratio (kg/kg).

\end{itemize}

\item[{Returns}] \leavevmode
\textbf{rl} --
Data for liquid water mixing ratio (kg/kg).

\item[{Return type}] \leavevmode
float or ndarray

\end{description}\end{quote}

\end{fulllineitems}

\index{rl\_from\_rt\_rv\_ri() (in module atmos.equations)}

\begin{fulllineitems}
\phantomsection\label{atmos:atmos.equations.rl_from_rt_rv_ri}\pysiglinewithargsret{\code{atmos.equations.}\bfcode{rl\_from\_rt\_rv\_ri}}{\emph{rt}, \emph{rv}, \emph{ri}}{}
Calculates liquid water mixing ratio (kg/kg).

\(r_l = r_t-r_v-r_i\)
\begin{quote}\begin{description}
\item[{Parameters}] \leavevmode\begin{itemize}
\item {} 
\textbf{rt} (\emph{float or ndarray}) -- Data for total water mixing ratio (kg/kg).

\item {} 
\textbf{rv} (\emph{float or ndarray}) -- Data for water vapor mixing ratio (kg/kg).

\item {} 
\textbf{ri} (\emph{float or ndarray}) -- Data for ice mixing ratio (kg/kg).

\end{itemize}

\item[{Returns}] \leavevmode
\textbf{rl} --
Data for liquid water mixing ratio (kg/kg).

\item[{Return type}] \leavevmode
float or ndarray

\end{description}\end{quote}

\end{fulllineitems}

\index{rt\_from\_ri\_rv\_rl() (in module atmos.equations)}

\begin{fulllineitems}
\phantomsection\label{atmos:atmos.equations.rt_from_ri_rv_rl}\pysiglinewithargsret{\code{atmos.equations.}\bfcode{rt\_from\_ri\_rv\_rl}}{\emph{ri}, \emph{rv}, \emph{rl}}{}
Calculates total water mixing ratio (kg/kg).

\(r_t = r_i+r_v+r_l\)
\begin{quote}\begin{description}
\item[{Parameters}] \leavevmode\begin{itemize}
\item {} 
\textbf{ri} (\emph{float or ndarray}) -- Data for ice mixing ratio (kg/kg).

\item {} 
\textbf{rv} (\emph{float or ndarray}) -- Data for water vapor mixing ratio (kg/kg).

\item {} 
\textbf{rl} (\emph{float or ndarray}) -- Data for liquid water mixing ratio (kg/kg).

\end{itemize}

\item[{Returns}] \leavevmode
\textbf{rt} --
Data for total water mixing ratio (kg/kg).

\item[{Return type}] \leavevmode
float or ndarray

\end{description}\end{quote}

\end{fulllineitems}

\index{rt\_from\_rv() (in module atmos.equations)}

\begin{fulllineitems}
\phantomsection\label{atmos:atmos.equations.rt_from_rv}\pysiglinewithargsret{\code{atmos.equations.}\bfcode{rt\_from\_rv}}{\emph{rv}}{}
Calculates total water mixing ratio (kg/kg) assuming liquid water can be
neglected and ice can be neglected.

\(r_t = r_v\)
\begin{quote}\begin{description}
\item[{Parameters}] \leavevmode
\textbf{rv} (\emph{float or ndarray}) -- Data for water vapor mixing ratio (kg/kg).

\item[{Returns}] \leavevmode
\textbf{rt} --
Data for total water mixing ratio (kg/kg).

\item[{Return type}] \leavevmode
float or ndarray

\end{description}\end{quote}

\end{fulllineitems}

\index{rt\_from\_rv\_ri() (in module atmos.equations)}

\begin{fulllineitems}
\phantomsection\label{atmos:atmos.equations.rt_from_rv_ri}\pysiglinewithargsret{\code{atmos.equations.}\bfcode{rt\_from\_rv\_ri}}{\emph{rv}, \emph{ri}}{}
Calculates total water mixing ratio (kg/kg) assuming liquid water can be
neglected.

\(r_t = r_v+r_l\)
\begin{quote}\begin{description}
\item[{Parameters}] \leavevmode\begin{itemize}
\item {} 
\textbf{rv} (\emph{float or ndarray}) -- Data for water vapor mixing ratio (kg/kg).

\item {} 
\textbf{ri} (\emph{float or ndarray}) -- Data for ice mixing ratio (kg/kg).

\end{itemize}

\item[{Returns}] \leavevmode
\textbf{rt} --
Data for total water mixing ratio (kg/kg).

\item[{Return type}] \leavevmode
float or ndarray

\end{description}\end{quote}

\end{fulllineitems}

\index{rt\_from\_rv\_rl() (in module atmos.equations)}

\begin{fulllineitems}
\phantomsection\label{atmos:atmos.equations.rt_from_rv_rl}\pysiglinewithargsret{\code{atmos.equations.}\bfcode{rt\_from\_rv\_rl}}{\emph{rv}, \emph{rl}}{}
Calculates total water mixing ratio (kg/kg) assuming ice can be neglected.

\(r_t = r_v+r_l\)
\begin{quote}\begin{description}
\item[{Parameters}] \leavevmode\begin{itemize}
\item {} 
\textbf{rv} (\emph{float or ndarray}) -- Data for water vapor mixing ratio (kg/kg).

\item {} 
\textbf{rl} (\emph{float or ndarray}) -- Data for liquid water mixing ratio (kg/kg).

\end{itemize}

\item[{Returns}] \leavevmode
\textbf{rt} --
Data for total water mixing ratio (kg/kg).

\item[{Return type}] \leavevmode
float or ndarray

\end{description}\end{quote}

\end{fulllineitems}

\index{rv\_from\_RH\_rvs() (in module atmos.equations)}

\begin{fulllineitems}
\phantomsection\label{atmos:atmos.equations.rv_from_RH_rvs}\pysiglinewithargsret{\code{atmos.equations.}\bfcode{rv\_from\_RH\_rvs}}{\emph{RH}, \emph{rvs}}{}
Calculates water vapor mixing ratio (kg/kg).

\(r_v = \frac{RH}{100} r_{vs}\)
\begin{quote}\begin{description}
\item[{Parameters}] \leavevmode\begin{itemize}
\item {} 
\textbf{RH} (\emph{float or ndarray}) -- Data for relative humidity (\%).

\item {} 
\textbf{rvs} (\emph{float or ndarray}) -- Data for saturation water vapor mixing ratio (kg/kg).

\end{itemize}

\item[{Returns}] \leavevmode
\textbf{rv} --
Data for water vapor mixing ratio (kg/kg).

\item[{Return type}] \leavevmode
float or ndarray

\end{description}\end{quote}

\end{fulllineitems}

\index{rv\_from\_p\_e() (in module atmos.equations)}

\begin{fulllineitems}
\phantomsection\label{atmos:atmos.equations.rv_from_p_e}\pysiglinewithargsret{\code{atmos.equations.}\bfcode{rv\_from\_p\_e}}{\emph{p}, \emph{e}}{}
Calculates water vapor mixing ratio (kg/kg).

\(rv = (\frac{Rd}{Rv}) \frac{e}{p-e}\)
\begin{quote}\begin{description}
\item[{Parameters}] \leavevmode\begin{itemize}
\item {} 
\textbf{p} (\emph{float or ndarray}) -- Data for pressure (Pa).

\item {} 
\textbf{e} (\emph{float or ndarray}) -- Data for water vapor partial pressure (Pa).

\end{itemize}

\item[{Returns}] \leavevmode
\textbf{rv} --
Data for water vapor mixing ratio (kg/kg).

\item[{Return type}] \leavevmode
float or ndarray

\end{description}\end{quote}

\end{fulllineitems}

\index{rv\_from\_qv() (in module atmos.equations)}

\begin{fulllineitems}
\phantomsection\label{atmos:atmos.equations.rv_from_qv}\pysiglinewithargsret{\code{atmos.equations.}\bfcode{rv\_from\_qv}}{\emph{qv}}{}
Calculates water vapor mixing ratio (kg/kg).

\(r_v = \frac{q_v}{1-q_v}\)
\begin{quote}\begin{description}
\item[{Parameters}] \leavevmode
\textbf{qv} (\emph{float or ndarray}) -- Data for specific humidity (kg/kg).

\item[{Returns}] \leavevmode
\textbf{rv} --
Data for water vapor mixing ratio (kg/kg).

\item[{Return type}] \leavevmode
float or ndarray

\end{description}\end{quote}

\end{fulllineitems}

\index{rv\_from\_qv\_lwv() (in module atmos.equations)}

\begin{fulllineitems}
\phantomsection\label{atmos:atmos.equations.rv_from_qv_lwv}\pysiglinewithargsret{\code{atmos.equations.}\bfcode{rv\_from\_qv\_lwv}}{\emph{qv}}{}
Calculates water vapor mixing ratio (kg/kg) assuming terms that are second-order
in moisture quantities can be neglected (eg. qv == rv).

\(r_v = q_v\)
\begin{quote}\begin{description}
\item[{Parameters}] \leavevmode
\textbf{qv} (\emph{float or ndarray}) -- Data for specific humidity (kg/kg).

\item[{Returns}] \leavevmode
\textbf{rv} --
Data for water vapor mixing ratio (kg/kg).

\item[{Return type}] \leavevmode
float or ndarray

\end{description}\end{quote}

\end{fulllineitems}

\index{rv\_from\_rt() (in module atmos.equations)}

\begin{fulllineitems}
\phantomsection\label{atmos:atmos.equations.rv_from_rt}\pysiglinewithargsret{\code{atmos.equations.}\bfcode{rv\_from\_rt}}{\emph{rt}}{}
Calculates water vapor mixing ratio (kg/kg) assuming liquid water can be
neglected and ice can be neglected.

\(r_v = r_t\)
\begin{quote}\begin{description}
\item[{Parameters}] \leavevmode
\textbf{rt} (\emph{float or ndarray}) -- Data for total water mixing ratio (kg/kg).

\item[{Returns}] \leavevmode
\textbf{rv} --
Data for water vapor mixing ratio (kg/kg).

\item[{Return type}] \leavevmode
float or ndarray

\end{description}\end{quote}

\end{fulllineitems}

\index{rv\_from\_rt\_ri() (in module atmos.equations)}

\begin{fulllineitems}
\phantomsection\label{atmos:atmos.equations.rv_from_rt_ri}\pysiglinewithargsret{\code{atmos.equations.}\bfcode{rv\_from\_rt\_ri}}{\emph{rt}, \emph{ri}}{}
Calculates water vapor mixing ratio (kg/kg) assuming liquid water can be
neglected.

\(r_v = r_t - r_i\)
\begin{quote}\begin{description}
\item[{Parameters}] \leavevmode\begin{itemize}
\item {} 
\textbf{rt} (\emph{float or ndarray}) -- Data for total water mixing ratio (kg/kg).

\item {} 
\textbf{ri} (\emph{float or ndarray}) -- Data for ice mixing ratio (kg/kg).

\end{itemize}

\item[{Returns}] \leavevmode
\textbf{rv} --
Data for water vapor mixing ratio (kg/kg).

\item[{Return type}] \leavevmode
float or ndarray

\end{description}\end{quote}

\end{fulllineitems}

\index{rv\_from\_rt\_rl() (in module atmos.equations)}

\begin{fulllineitems}
\phantomsection\label{atmos:atmos.equations.rv_from_rt_rl}\pysiglinewithargsret{\code{atmos.equations.}\bfcode{rv\_from\_rt\_rl}}{\emph{rt}, \emph{rl}}{}
Calculates water vapor mixing ratio (kg/kg) assuming ice can be neglected.

\(r_v = r_t-r_l\)
\begin{quote}\begin{description}
\item[{Parameters}] \leavevmode\begin{itemize}
\item {} 
\textbf{rt} (\emph{float or ndarray}) -- Data for total water mixing ratio (kg/kg).

\item {} 
\textbf{rl} (\emph{float or ndarray}) -- Data for liquid water mixing ratio (kg/kg).

\end{itemize}

\item[{Returns}] \leavevmode
\textbf{rv} --
Data for water vapor mixing ratio (kg/kg).

\item[{Return type}] \leavevmode
float or ndarray

\end{description}\end{quote}

\end{fulllineitems}

\index{rv\_from\_rt\_rl\_ri() (in module atmos.equations)}

\begin{fulllineitems}
\phantomsection\label{atmos:atmos.equations.rv_from_rt_rl_ri}\pysiglinewithargsret{\code{atmos.equations.}\bfcode{rv\_from\_rt\_rl\_ri}}{\emph{rt}, \emph{rl}, \emph{ri}}{}
Calculates water vapor mixing ratio (kg/kg).

\(r_v = r_t-r_l-r_i\)
\begin{quote}\begin{description}
\item[{Parameters}] \leavevmode\begin{itemize}
\item {} 
\textbf{rt} (\emph{float or ndarray}) -- Data for total water mixing ratio (kg/kg).

\item {} 
\textbf{rl} (\emph{float or ndarray}) -- Data for liquid water mixing ratio (kg/kg).

\item {} 
\textbf{ri} (\emph{float or ndarray}) -- Data for ice mixing ratio (kg/kg).

\end{itemize}

\item[{Returns}] \leavevmode
\textbf{rv} --
Data for water vapor mixing ratio (kg/kg).

\item[{Return type}] \leavevmode
float or ndarray

\end{description}\end{quote}

\end{fulllineitems}

\index{rvs\_from\_p\_es() (in module atmos.equations)}

\begin{fulllineitems}
\phantomsection\label{atmos:atmos.equations.rvs_from_p_es}\pysiglinewithargsret{\code{atmos.equations.}\bfcode{rvs\_from\_p\_es}}{\emph{p}, \emph{es}}{}
Calculates saturation water vapor mixing ratio (kg/kg).

\(r_{vs} = rv\_from\_p\_e(p, e_s)\)
\begin{quote}\begin{description}
\item[{Parameters}] \leavevmode\begin{itemize}
\item {} 
\textbf{p} (\emph{float or ndarray}) -- Data for pressure (Pa).

\item {} 
\textbf{es} (\emph{float or ndarray}) -- Data for saturation water vapor partial pressure (Pa).

\end{itemize}

\item[{Returns}] \leavevmode
\textbf{rvs} --
Data for saturation water vapor mixing ratio (kg/kg).

\item[{Return type}] \leavevmode
float or ndarray

\end{description}\end{quote}

\end{fulllineitems}

\index{rvs\_from\_qvs() (in module atmos.equations)}

\begin{fulllineitems}
\phantomsection\label{atmos:atmos.equations.rvs_from_qvs}\pysiglinewithargsret{\code{atmos.equations.}\bfcode{rvs\_from\_qvs}}{\emph{qvs}}{}
Calculates saturation water vapor mixing ratio (kg/kg).

\(r_v = rv\_from\_qv(q_{vs})\)
\begin{quote}\begin{description}
\item[{Parameters}] \leavevmode
\textbf{qvs} (\emph{float or ndarray}) -- Data for saturation specific humidity (kg/kg).

\item[{Returns}] \leavevmode
\textbf{rvs} --
Data for saturation water vapor mixing ratio (kg/kg).

\item[{Return type}] \leavevmode
float or ndarray

\end{description}\end{quote}

\end{fulllineitems}

\index{rvs\_from\_qvs\_lwv() (in module atmos.equations)}

\begin{fulllineitems}
\phantomsection\label{atmos:atmos.equations.rvs_from_qvs_lwv}\pysiglinewithargsret{\code{atmos.equations.}\bfcode{rvs\_from\_qvs\_lwv}}{\emph{qvs}}{}
Calculates saturation water vapor mixing ratio (kg/kg) assuming terms that are
second-order in moisture quantities can be neglected (eg. qv == rv).

\(r_v = rv\_from\_qv(q_{vs})\)
\begin{quote}\begin{description}
\item[{Parameters}] \leavevmode
\textbf{qvs} (\emph{float or ndarray}) -- Data for saturation specific humidity (kg/kg).

\item[{Returns}] \leavevmode
\textbf{rvs} --
Data for saturation water vapor mixing ratio (kg/kg).

\item[{Return type}] \leavevmode
float or ndarray

\end{description}\end{quote}

\end{fulllineitems}

\index{theta\_from\_p\_T() (in module atmos.equations)}

\begin{fulllineitems}
\phantomsection\label{atmos:atmos.equations.theta_from_p_T}\pysiglinewithargsret{\code{atmos.equations.}\bfcode{theta\_from\_p\_T}}{\emph{p}, \emph{T}}{}
Calculates potential temperature (K) assuming Cp is constant and equal to Cp for
dry air at 0C.

\(\theta = T (\frac{10^5}{p})^(\frac{R_d}{C_{pd}})\)
\begin{quote}\begin{description}
\item[{Parameters}] \leavevmode\begin{itemize}
\item {} 
\textbf{p} (\emph{float or ndarray}) -- Data for pressure (Pa).

\item {} 
\textbf{T} (\emph{float or ndarray}) -- Data for temperature (K).

\end{itemize}

\item[{Returns}] \leavevmode
\textbf{theta} --
Data for potential temperature (K).

\item[{Return type}] \leavevmode
float or ndarray

\end{description}\end{quote}

\end{fulllineitems}

\index{thetae\_from\_T\_RH\_rv\_lwv() (in module atmos.equations)}

\begin{fulllineitems}
\phantomsection\label{atmos:atmos.equations.thetae_from_T_RH_rv_lwv}\pysiglinewithargsret{\code{atmos.equations.}\bfcode{thetae\_from\_T\_RH\_rv\_lwv}}{\emph{T}, \emph{RH}, \emph{rv}}{}
Calculates equivalent temperature (K) assuming terms that are second-order in
moisture quantities can be neglected (eg. qv == rv).

\(\theta_e = T*(\frac{10^5}{p})^(\frac{R_d}{C_{pd}}) RH^{-r_v
\frac{Rv}{C_{pd}}} exp(L_v \frac{Rv}{C_{pd}})\)
\begin{quote}\begin{description}
\item[{Parameters}] \leavevmode\begin{itemize}
\item {} 
\textbf{T} (\emph{float or ndarray}) -- Data for temperature (K).

\item {} 
\textbf{RH} (\emph{float or ndarray}) -- Data for relative humidity (\%).

\item {} 
\textbf{rv} (\emph{float or ndarray}) -- Data for water vapor mixing ratio (kg/kg).

\end{itemize}

\item[{Returns}] \leavevmode
\textbf{thetae} --
Data for equivalent temperature (K).

\item[{Return type}] \leavevmode
float or ndarray

\end{description}\end{quote}

\end{fulllineitems}

\index{thetae\_from\_p\_T\_Tlcl\_rv\_Bolton() (in module atmos.equations)}

\begin{fulllineitems}
\phantomsection\label{atmos:atmos.equations.thetae_from_p_T_Tlcl_rv_Bolton}\pysiglinewithargsret{\code{atmos.equations.}\bfcode{thetae\_from\_p\_T\_Tlcl\_rv\_Bolton}}{\emph{p}, \emph{T}, \emph{Tlcl}, \emph{rv}}{}
Calculates equivalent temperature (K) assuming the assumptions in Bolton (1980)
hold, Cp is constant and equal to Cp for dry air at 0C, and liquid water can be
neglected.

\(\theta_e = T (\frac{10^5}{p})^(\frac{R_d}{C_{pd}})(1-0.28 r_v))
exp((\frac{3.376}{T_{lcl}}-0.00254) (r_v \times 10^3) (1+0.81 r_v))\)
\begin{quote}\begin{description}
\item[{Parameters}] \leavevmode\begin{itemize}
\item {} 
\textbf{p} (\emph{float or ndarray}) -- Data for pressure (Pa).

\item {} 
\textbf{T} (\emph{float or ndarray}) -- Data for temperature (K).

\item {} 
\textbf{Tlcl} (\emph{float or ndarray}) -- Data for temperature at lifting condensation level (K).

\item {} 
\textbf{rv} (\emph{float or ndarray}) -- Data for water vapor mixing ratio (kg/kg).

\end{itemize}

\item[{Returns}] \leavevmode
\textbf{thetae} --
Data for equivalent temperature (K).

\item[{Return type}] \leavevmode
float or ndarray

\end{description}\end{quote}
\paragraph{Notes}

This is one of the most accurate ways of computing thetae, with an
error of less than 0.2K due mainly to assuming Cp does not vary with
temperature or pressure.

\textbf{References}
\begin{description}
\item[{Bolton, D. 1980: The Computation of Equivalent Potential Temperature.}] \leavevmode
Mon. Wea. Rev., 108, 1046–1053.
doi: \href{http://dx.doi.org/10.1175/1520-0493(1980}{http://dx.doi.org/10.1175/1520-0493(1980})108\textless{}1046:TCOEPT\textgreater{}2.0.CO;2

\item[{Davies-Jones, R. 2009: On Formulas for Equivalent Potential}] \leavevmode
Temperature. Mon. Wea. Rev., 137, 3137–3148.
doi: \href{http://dx.doi.org/10.1175/2009MWR2774.1}{http://dx.doi.org/10.1175/2009MWR2774.1}

\end{description}

\end{fulllineitems}

\index{thetae\_from\_p\_e\_T\_RH\_rv\_rt() (in module atmos.equations)}

\begin{fulllineitems}
\phantomsection\label{atmos:atmos.equations.thetae_from_p_e_T_RH_rv_rt}\pysiglinewithargsret{\code{atmos.equations.}\bfcode{thetae\_from\_p\_e\_T\_RH\_rv\_rt}}{\emph{p}, \emph{e}, \emph{T}, \emph{RH}, \emph{rv}, \emph{rt}}{}
Calculates equivalent temperature (K).

\(\theta_e = T (\frac{10^5}{p})^(\frac{R_d}{C_{pd}) + r_t C_l}) RH^{-r_v
\frac{R_v}{C_{pd} +r_t C_l}} exp(L_v \frac{r_v}{C_{pd}+r_t C_l})\)
\begin{quote}\begin{description}
\item[{Parameters}] \leavevmode\begin{itemize}
\item {} 
\textbf{p} (\emph{float or ndarray}) -- Data for pressure (Pa).

\item {} 
\textbf{e} (\emph{float or ndarray}) -- Data for water vapor partial pressure (Pa).

\item {} 
\textbf{T} (\emph{float or ndarray}) -- Data for temperature (K).

\item {} 
\textbf{RH} (\emph{float or ndarray}) -- Data for relative humidity (\%).

\item {} 
\textbf{rv} (\emph{float or ndarray}) -- Data for water vapor mixing ratio (kg/kg).

\item {} 
\textbf{rt} (\emph{float or ndarray}) -- Data for total water mixing ratio (kg/kg).

\end{itemize}

\item[{Returns}] \leavevmode
\textbf{thetae} --
Data for equivalent temperature (K).

\item[{Return type}] \leavevmode
float or ndarray

\end{description}\end{quote}
\paragraph{Notes}

\textbf{References}
\begin{description}
\item[{American Meteorological Society Glossary of Meteorology}] \leavevmode
\href{http://glossary.ametsoc.org/wiki/Equivalent\_potential\_temperature}{http://glossary.ametsoc.org/wiki/Equivalent\_potential\_temperature}
Retrieved April 23, 2015

\end{description}

\end{fulllineitems}

\index{thetaes\_from\_p\_T\_rvs\_Bolton() (in module atmos.equations)}

\begin{fulllineitems}
\phantomsection\label{atmos:atmos.equations.thetaes_from_p_T_rvs_Bolton}\pysiglinewithargsret{\code{atmos.equations.}\bfcode{thetaes\_from\_p\_T\_rvs\_Bolton}}{\emph{p}, \emph{T}, \emph{rvs}}{}
Calculates saturation equivalent temperature (K) assuming the assumptions in
Bolton (1980) hold and Cp is constant and equal to Cp for dry air at 0C.

\(\theta_{es} = thetae\_from\_p\_T\_Tlcl\_rv\_Bolton(p, T, T, r_{vs})\)
\begin{quote}\begin{description}
\item[{Parameters}] \leavevmode\begin{itemize}
\item {} 
\textbf{p} (\emph{float or ndarray}) -- Data for pressure (Pa).

\item {} 
\textbf{T} (\emph{float or ndarray}) -- Data for temperature (K).

\item {} 
\textbf{rvs} (\emph{float or ndarray}) -- Data for saturation water vapor mixing ratio (kg/kg).

\end{itemize}

\item[{Returns}] \leavevmode
\textbf{thetaes} --
Data for saturation equivalent temperature (K).

\item[{Return type}] \leavevmode
float or ndarray

\end{description}\end{quote}
\paragraph{Notes}

See thetae\_from\_theta\_Tlcl\_rv\_Bolton for more information.

\textbf{References}
\begin{description}
\item[{Bolton, D. 1980: The Computation of Equivalent Potential Temperature.}] \leavevmode
Mon. Wea. Rev., 108, 1046–1053.
doi: \href{http://dx.doi.org/10.1175/1520-0493(1980}{http://dx.doi.org/10.1175/1520-0493(1980})108\textless{}1046:TCOEPT\textgreater{}2.0.CO;2

\item[{Davies-Jones, R. 2009: On Formulas for Equivalent Potential}] \leavevmode
Temperature. Mon. Wea. Rev., 137, 3137–3148.
doi: \href{http://dx.doi.org/10.1175/2009MWR2774.1}{http://dx.doi.org/10.1175/2009MWR2774.1}

\end{description}

\end{fulllineitems}

\index{w\_from\_omega\_rho\_hydrostatic() (in module atmos.equations)}

\begin{fulllineitems}
\phantomsection\label{atmos:atmos.equations.w_from_omega_rho_hydrostatic}\pysiglinewithargsret{\code{atmos.equations.}\bfcode{w\_from\_omega\_rho\_hydrostatic}}{\emph{omega}, \emph{rho}}{}
Calculates vertical velocity (m/s) assuming g is constant and hydrostatic
balance.

\(w = - \frac{\omega}{\rho g_0}\)
\begin{quote}\begin{description}
\item[{Parameters}] \leavevmode\begin{itemize}
\item {} 
\textbf{omega} (\emph{float or ndarray}) -- Data for vertical velocity expressed as tendency of pressure (Pa/s).

\item {} 
\textbf{rho} (\emph{float or ndarray}) -- Data for density (kg/m\textasciicircum{}3).

\end{itemize}

\item[{Returns}] \leavevmode
\textbf{w} --
Data for vertical velocity (m/s).

\item[{Return type}] \leavevmode
float or ndarray

\end{description}\end{quote}

\end{fulllineitems}

\index{z\_from\_Phi() (in module atmos.equations)}

\begin{fulllineitems}
\phantomsection\label{atmos:atmos.equations.z_from_Phi}\pysiglinewithargsret{\code{atmos.equations.}\bfcode{z\_from\_Phi}}{\emph{Phi}}{}
Calculates height (m) assuming g is constant.

\(z = \frac{\Phi}{g_0}\)
\begin{quote}\begin{description}
\item[{Parameters}] \leavevmode
\textbf{Phi} (\emph{float or ndarray}) -- Data for geopotential (m\textasciicircum{}2/s\textasciicircum{}2).

\item[{Returns}] \leavevmode
\textbf{z} --
Data for height (m).

\item[{Return type}] \leavevmode
float or ndarray

\end{description}\end{quote}

\end{fulllineitems}



\section{atmos.solve module}
\label{atmos:module-atmos.solve}\label{atmos:atmos-solve-module}\index{atmos.solve (module)}\begin{description}
\item[{solve.py: Utilities that use equations to solve for quantities, given other}] \leavevmode
quantities and a set of assumptions.

\end{description}
\index{BaseSolver (class in atmos.solve)}

\begin{fulllineitems}
\phantomsection\label{atmos:atmos.solve.BaseSolver}\pysiglinewithargsret{\strong{class }\code{atmos.solve.}\bfcode{BaseSolver}}{\emph{**kwargs}}{}
Bases: \code{object}

Base class for solving systems of equations. Should not be instantiated,
as it is not associated with any equations.

Initializes with the given assumptions enabled, and variables passed as
keyword arguments stored.
\begin{quote}\begin{description}
\item[{Parameters}] \leavevmode\begin{itemize}
\item {} 
\textbf{assumptions} (\emph{tuple, optional}) -- Strings specifying which assumptions to enable. Overrides the default
assumptions. See below for a list of default assumptions.

\item {} 
\textbf{add\_assumptions} (\emph{tuple, optional}) -- Strings specifying assumptions to use in addition to the default
assumptions. May not be given in combination with the assumptions kwarg.

\item {} 
\textbf{remove\_assumptions} (\emph{tuple, optional}) -- Strings specifying assumptions not to use from the default assumptions.
May not be given in combination with the assumptions kwarg. May not
contain strings that are contained in add\_assumptions, if given.

\item {} 
\textbf{**kwargs} -- 
Keyword arguments used to pass in arrays of data that correspond to
quantities used for calculations. For a complete list of kwargs that
may be used, see the Quantity Parameters section below.


\end{itemize}

\item[{Returns}] \leavevmode
\textbf{out} --
A BaseSolver object with the specified assumptions and variables.

\item[{Return type}] \leavevmode
BaseSolver

\end{description}\end{quote}
\paragraph{Notes}

\textbf{Quantity kwargs}

\textless{}quantity parameter list goes here\textgreater{}

\textbf{Assumptions}

\textless{}default assumptions list goes here\textgreater{}

\textbf{Assumption descriptions}

\textless{}assumptions list goes here\textgreater{}
\index{calculate() (atmos.solve.BaseSolver method)}

\begin{fulllineitems}
\phantomsection\label{atmos:atmos.solve.BaseSolver.calculate}\pysiglinewithargsret{\bfcode{calculate}}{\emph{*args}}{}
Calculates and returns a requested quantity from quantities stored in this
object at initialization.
\begin{quote}\begin{description}
\item[{Parameters}] \leavevmode
\textbf{varname\_out} (\emph{string}) -- Name of quantity to be calculated.

\item[{Returns}] \leavevmode
\textbf{quantity} --
Calculated quantity, in units listed under quantity parameters.

\item[{Return type}] \leavevmode
ndarray

\end{description}\end{quote}
\paragraph{Notes}

See the documentation for this object for a complete list of quantities
that may be calculated, in the ``Quantity Parameters'' section.
\begin{quote}\begin{description}
\item[{Raises}] \leavevmode
\code{ValueError} --
If the output quantity cannot be determined from the input
quantities.

\end{description}\end{quote}
\paragraph{Examples}

Calculating pressure from virtual temperature and density:

\begin{Verbatim}[commandchars=\\\{\}]
\PYG{g+gp}{\PYGZgt{}\PYGZgt{}\PYGZgt{} }\PYG{n}{solver} \PYG{o}{=} \PYG{n}{FluidSolver}\PYG{p}{(}\PYG{n}{Tv}\PYG{o}{=}\PYG{l+m+mf}{273.}\PYG{p}{,} \PYG{n}{rho}\PYG{o}{=}\PYG{l+m+mf}{1.27}\PYG{p}{)}
\PYG{g+gp}{\PYGZgt{}\PYGZgt{}\PYGZgt{} }\PYG{n}{solver}\PYG{o}{.}\PYG{n}{calculate}\PYG{p}{(}\PYG{l+s}{\PYGZsq{}}\PYG{l+s}{p}\PYG{l+s}{\PYGZsq{}}\PYG{p}{)}
\PYG{g+go}{99519.638400000011}
\end{Verbatim}

Same calculation, but also returning a list of functions used:

\begin{Verbatim}[commandchars=\\\{\}]
\PYG{g+gp}{\PYGZgt{}\PYGZgt{}\PYGZgt{} }\PYG{n}{solver} \PYG{o}{=} \PYG{n}{FluidSolver}\PYG{p}{(}\PYG{n}{Tv}\PYG{o}{=}\PYG{l+m+mf}{273.}\PYG{p}{,} \PYG{n}{rho}\PYG{o}{=}\PYG{l+m+mf}{1.27}\PYG{p}{,} \PYG{n}{debug}\PYG{o}{=}\PYG{n+nb+bp}{True}\PYG{p}{)}
\PYG{g+gp}{\PYGZgt{}\PYGZgt{}\PYGZgt{} }\PYG{n}{p}\PYG{p}{,} \PYG{n}{funcs} \PYG{o}{=} \PYG{n}{solver}\PYG{o}{.}\PYG{n}{calculate}\PYG{p}{(}\PYG{l+s}{\PYGZsq{}}\PYG{l+s}{p}\PYG{l+s}{\PYGZsq{}}\PYG{p}{)}
\PYG{g+gp}{\PYGZgt{}\PYGZgt{}\PYGZgt{} }\PYG{n}{funcs}
\PYG{g+go}{(\PYGZlt{}function atmos.equations.p\PYGZus{}from\PYGZus{}rho\PYGZus{}Tv\PYGZus{}ideal\PYGZus{}gas\PYGZgt{},)}
\end{Verbatim}

Same calculation with temperature instead, ignoring virtual temperature
correction:

\begin{Verbatim}[commandchars=\\\{\}]
\PYG{g+gp}{\PYGZgt{}\PYGZgt{}\PYGZgt{} }\PYG{n}{solver} \PYG{o}{=} \PYG{n}{FluidSolver}\PYG{p}{(}\PYG{n}{T}\PYG{o}{=}\PYG{l+m+mf}{273.}\PYG{p}{,} \PYG{n}{rho}\PYG{o}{=}\PYG{l+m+mf}{1.27}\PYG{p}{,} \PYG{n}{add\PYGZus{}assumptions}\PYG{o}{=}\PYG{p}{(}\PYG{l+s}{\PYGZsq{}}\PYG{l+s}{Tv equals T}\PYG{l+s}{\PYGZsq{}}\PYG{p}{,}\PYG{p}{)}\PYG{p}{)}
\PYG{g+gp}{\PYGZgt{}\PYGZgt{}\PYGZgt{} }\PYG{n}{solver}\PYG{o}{.}\PYG{n}{calculate}\PYG{p}{(}\PYG{l+s}{\PYGZsq{}}\PYG{l+s}{p}\PYG{l+s}{\PYGZsq{}}\PYG{p}{,}\PYG{p}{)}
\PYG{g+go}{99519.638400000011}
\end{Verbatim}

\end{fulllineitems}


\end{fulllineitems}

\index{ExcludeError}

\begin{fulllineitems}
\phantomsection\label{atmos:atmos.solve.ExcludeError}\pysigline{\strong{exception }\code{atmos.solve.}\bfcode{ExcludeError}}
Bases: \code{exceptions.Exception}

\end{fulllineitems}

\index{FluidSolver (class in atmos.solve)}

\begin{fulllineitems}
\phantomsection\label{atmos:atmos.solve.FluidSolver}\pysiglinewithargsret{\strong{class }\code{atmos.solve.}\bfcode{FluidSolver}}{\emph{**kwargs}}{}
Bases: {\hyperref[atmos:atmos.solve.BaseSolver]{\code{atmos.solve.BaseSolver}}}

Initializes with the given assumptions enabled, and variables passed as
keyword arguments stored.
\begin{quote}\begin{description}
\item[{Parameters}] \leavevmode\begin{itemize}
\item {} 
\textbf{assumptions} (\emph{tuple, optional}) -- Strings specifying which assumptions to enable. Overrides the default
assumptions. See below for a list of default assumptions.

\item {} 
\textbf{add\_assumptions} (\emph{tuple, optional}) -- Strings specifying assumptions to use in addition to the default
assumptions. May not be given in combination with the assumptions kwarg.

\item {} 
\textbf{remove\_assumptions} (\emph{tuple, optional}) -- Strings specifying assumptions not to use from the default assumptions.
May not be given in combination with the assumptions kwarg. May not
contain strings that are contained in add\_assumptions, if given.

\item {} 
\textbf{**kwargs} -- 
Keyword arguments used to pass in arrays of data that correspond to
quantities used for calculations. For a complete list of kwargs that
may be used, see the Quantity Parameters section below.


\end{itemize}

\item[{Returns}] \leavevmode
\textbf{out} --
A FluidSolver object with the specified assumptions and variables.

\item[{Return type}] \leavevmode
FluidSolver

\end{description}\end{quote}
\paragraph{Notes}

\textbf{Quantity kwargs}
\begin{itemize}
\item {} 
\textbf{AH} -- absolute humidity (kg/m\textasciicircum{}3)

\item {} 
\textbf{DSE} -- dry static energy (J)

\item {} 
\textbf{e} -- water vapor partial pressure (Pa)

\item {} 
\textbf{es} -- saturation water vapor partial pressure (Pa)

\item {} 
\textbf{f} -- Coriolis parameter (Hz)

\item {} 
\textbf{Gammam} -- moist adiabatic lapse rate (K/m)

\item {} 
\textbf{lat} -- latitude (degrees N)

\item {} 
\textbf{lon} -- longitude (degrees E)

\item {} 
\textbf{MSE} -- moist static energy (J)

\item {} 
\textbf{N2} -- squared Brunt-Vaisala frequency (Hz\textasciicircum{}2)

\item {} 
\textbf{omega} -- vertical velocity expressed as tendency of pressure (Pa/s)

\item {} 
\textbf{p} -- pressure (Pa)

\item {} 
\textbf{Phi} -- geopotential (m\textasciicircum{}2/s\textasciicircum{}2)

\item {} 
\textbf{plcl} -- pressure at lifting condensation level (Pa)

\item {} 
\textbf{qi} -- specific humidity with respect to ice (kg/kg)

\item {} 
\textbf{ql} -- specific humidity with respect to liquid water (kg/kg)

\item {} 
\textbf{qt} -- specific humidity with respect to total water (kg/kg)

\item {} 
\textbf{qv} -- specific humidity (kg/kg)

\item {} 
\textbf{qvs} -- saturation specific humidity (kg/kg)

\item {} 
\textbf{RB} -- bulk Richardson number (unitless)

\item {} 
\textbf{RH} -- relative humidity (\%)

\item {} 
\textbf{rho} -- density (kg/m\textasciicircum{}3)

\item {} 
\textbf{ri} -- ice mixing ratio (kg/kg)

\item {} 
\textbf{rl} -- liquid water mixing ratio (kg/kg)

\item {} 
\textbf{rt} -- total water mixing ratio (kg/kg)

\item {} 
\textbf{rv} -- water vapor mixing ratio (kg/kg)

\item {} 
\textbf{rvs} -- saturation water vapor mixing ratio (kg/kg)

\item {} 
\textbf{T} -- temperature (K)

\item {} 
\textbf{Td} -- dewpoint temperature (K)

\item {} 
\textbf{theta} -- potential temperature (K)

\item {} 
\textbf{thetae} -- equivalent temperature (K)

\item {} 
\textbf{thetaes} -- saturation equivalent temperature (K)

\item {} 
\textbf{Tlcl} -- temperature at lifting condensation level (K)

\item {} 
\textbf{Tv} -- virtual temperature (K)

\item {} 
\textbf{Tw} -- wet bulb temperature (K)

\item {} 
\textbf{u} -- eastward zonal wind velocity (m/s)

\item {} 
\textbf{v} -- northward meridional wind velocity (m/s)

\item {} 
\textbf{w} -- vertical velocity (m/s)

\item {} 
\textbf{x} -- x (m)

\item {} 
\textbf{y} -- y (m)

\item {} 
\textbf{Z} -- geopotential height (m)

\item {} 
\textbf{z} -- height (m)

\end{itemize}

\textbf{Assumptions}

Default assumptions are `ideal gas', `hydrostatic', `constant g', `constant Lv',
`constant Cp', `no liquid water', `no ice', `bolton'.

\textbf{Assumption descriptions}
\begin{itemize}
\item {} 
\textbf{bolton} -- the assumptions in Bolton (1980) hold

\item {} 
\textbf{constant Cp} -- Cp is constant and equal to Cp for dry air at 0C

\item {} 
\textbf{constant g} -- g is constant

\item {} 
\textbf{constant Lv} -- latent heat of vaporization of water is constant

\item {} 
\textbf{frozen bulb} -- the bulb is frozen

\item {} 
\textbf{goff-gratch} -- the Goff-Gratch equation for es

\item {} 
\textbf{hydrostatic} -- hydrostatic balance

\item {} 
\textbf{ideal gas} -- the ideal gas law holds

\item {} 
\textbf{low water vapor} -- terms that are second-order in moisture quantities can be neglected (eg. qv == rv)

\item {} 
\textbf{no ice} -- ice can be neglected

\item {} 
\textbf{no liquid water} -- liquid water can be neglected

\item {} 
\textbf{Tv equals T} -- the virtual temperature correction can be neglected

\item {} 
\textbf{unfrozen bulb} -- the bulb is not frozen

\end{itemize}
\paragraph{Examples}

Calculating pressure from virtual temperature and density:

\begin{Verbatim}[commandchars=\\\{\}]
\PYG{g+gp}{\PYGZgt{}\PYGZgt{}\PYGZgt{} }\PYG{n}{solver} \PYG{o}{=} \PYG{n}{FluidSolver}\PYG{p}{(}\PYG{n}{Tv}\PYG{o}{=}\PYG{l+m+mf}{273.}\PYG{p}{,} \PYG{n}{rho}\PYG{o}{=}\PYG{l+m+mf}{1.27}\PYG{p}{)}
\PYG{g+gp}{\PYGZgt{}\PYGZgt{}\PYGZgt{} }\PYG{n}{solver}\PYG{o}{.}\PYG{n}{calculate}\PYG{p}{(}\PYG{l+s}{\PYGZsq{}}\PYG{l+s}{p}\PYG{l+s}{\PYGZsq{}}\PYG{p}{)}
\PYG{g+go}{99519.638400000011}
\end{Verbatim}

Same calculation, but also returning a list of functions used:

\begin{Verbatim}[commandchars=\\\{\}]
\PYG{g+gp}{\PYGZgt{}\PYGZgt{}\PYGZgt{} }\PYG{n}{solver} \PYG{o}{=} \PYG{n}{FluidSolver}\PYG{p}{(}\PYG{n}{Tv}\PYG{o}{=}\PYG{l+m+mf}{273.}\PYG{p}{,} \PYG{n}{rho}\PYG{o}{=}\PYG{l+m+mf}{1.27}\PYG{p}{,} \PYG{n}{debug}\PYG{o}{=}\PYG{n+nb+bp}{True}\PYG{p}{)}
\PYG{g+gp}{\PYGZgt{}\PYGZgt{}\PYGZgt{} }\PYG{n}{p}\PYG{p}{,} \PYG{n}{funcs} \PYG{o}{=} \PYG{n}{solver}\PYG{o}{.}\PYG{n}{calculate}\PYG{p}{(}\PYG{l+s}{\PYGZsq{}}\PYG{l+s}{p}\PYG{l+s}{\PYGZsq{}}\PYG{p}{)}
\PYG{g+gp}{\PYGZgt{}\PYGZgt{}\PYGZgt{} }\PYG{n}{funcs}
\PYG{g+go}{(\PYGZlt{}function atmos.equations.p\PYGZus{}from\PYGZus{}rho\PYGZus{}Tv\PYGZus{}ideal\PYGZus{}gas\PYGZgt{},)}
\end{Verbatim}

Same calculation with temperature instead, ignoring virtual temperature
correction:

\begin{Verbatim}[commandchars=\\\{\}]
\PYG{g+gp}{\PYGZgt{}\PYGZgt{}\PYGZgt{} }\PYG{n}{solver} \PYG{o}{=} \PYG{n}{FluidSolver}\PYG{p}{(}\PYG{n}{T}\PYG{o}{=}\PYG{l+m+mf}{273.}\PYG{p}{,} \PYG{n}{rho}\PYG{o}{=}\PYG{l+m+mf}{1.27}\PYG{p}{,} \PYG{n}{add\PYGZus{}assumptions}\PYG{o}{=}\PYG{p}{(}\PYG{l+s}{\PYGZsq{}}\PYG{l+s}{Tv equals T}\PYG{l+s}{\PYGZsq{}}\PYG{p}{,}\PYG{p}{)}\PYG{p}{)}
\PYG{g+gp}{\PYGZgt{}\PYGZgt{}\PYGZgt{} }\PYG{n}{solver}\PYG{o}{.}\PYG{n}{calculate}\PYG{p}{(}\PYG{l+s}{\PYGZsq{}}\PYG{l+s}{p}\PYG{l+s}{\PYGZsq{}}\PYG{p}{,}\PYG{p}{)}
\PYG{g+go}{99519.638400000011}
\end{Verbatim}
\index{all\_assumptions (atmos.solve.FluidSolver attribute)}

\begin{fulllineitems}
\phantomsection\label{atmos:atmos.solve.FluidSolver.all_assumptions}\pysigline{\bfcode{all\_assumptions}\strong{ = (u'constant g', u'frozen bulb', u'unfrozen bulb', u'no ice', u'goff-gratch', u'ideal gas', u'no liquid water', u'constant Cp', u'Tv equals T', u'hydrostatic', u'bolton', u'constant Lv', u'low water vapor')}}
\end{fulllineitems}

\index{default\_assumptions (atmos.solve.FluidSolver attribute)}

\begin{fulllineitems}
\phantomsection\label{atmos:atmos.solve.FluidSolver.default_assumptions}\pysigline{\bfcode{default\_assumptions}\strong{ = (u'ideal gas', u'hydrostatic', u'constant g', u'constant Lv', u'constant Cp', u'no liquid water', u'no ice', u'bolton')}}
\end{fulllineitems}


\end{fulllineitems}

\index{SolverMeta (class in atmos.solve)}

\begin{fulllineitems}
\phantomsection\label{atmos:atmos.solve.SolverMeta}\pysigline{\strong{class }\code{atmos.solve.}\bfcode{SolverMeta}}
Bases: \code{type}

Metaclass for BaseSolver to automatically generate docstrings and assumption
lists for subclasses of BaseSolver.

\end{fulllineitems}

\index{calculate() (in module atmos.solve)}

\begin{fulllineitems}
\phantomsection\label{atmos:atmos.solve.calculate}\pysiglinewithargsret{\code{atmos.solve.}\bfcode{calculate}}{\emph{*args}, \emph{**kwargs}}{}
Calculates and returns a requested quantity from quantities passed in as
keyword arguments.
\begin{quote}\begin{description}
\item[{Parameters}] \leavevmode\begin{itemize}
\item {} 
\textbf{*args} (\emph{string}) -- Names of quantities to be calculated.

\item {} 
\textbf{assumptions} (\emph{tuple, optional}) -- Strings specifying which assumptions to enable. Overrides the default
assumptions. See below for a list of default assumptions.

\item {} 
\textbf{add\_assumptions} (\emph{tuple, optional}) -- Strings specifying assumptions to use in addition to the default
assumptions. May not be given in combination with the assumptions kwarg.

\item {} 
\textbf{remove\_assumptions} (\emph{tuple, optional}) -- Strings specifying assumptions not to use from the default assumptions.
May not be given in combination with the assumptions kwarg. May not
contain strings that are contained in add\_assumptions, if given.

\item {} 
\textbf{**kwargs} (\emph{ndarray, optional}) -- Keyword arguments used to pass in arrays of data that correspond to
quantities used for calculations. For a complete list of kwargs that
may be used, see the Quantity Parameters section below.

\end{itemize}

\item[{Returns}] \leavevmode
\textbf{quantity} --
Calculated quantity.
Return type is the same as quantity parameter types.
If multiple quantities are requested, returns a tuple containing the
quantities.

\item[{Return type}] \leavevmode
ndarray

\end{description}\end{quote}
\paragraph{Notes}

Calculating multiple quantities at once can avoid re-computing intermediate
quantities, but requires more memory.

\textbf{Quantity kwargs}
\begin{itemize}
\item {} 
\textbf{AH} -- absolute humidity (kg/m\textasciicircum{}3)

\item {} 
\textbf{DSE} -- dry static energy (J)

\item {} 
\textbf{e} -- water vapor partial pressure (Pa)

\item {} 
\textbf{es} -- saturation water vapor partial pressure (Pa)

\item {} 
\textbf{f} -- Coriolis parameter (Hz)

\item {} 
\textbf{Gammam} -- moist adiabatic lapse rate (K/m)

\item {} 
\textbf{lat} -- latitude (degrees N)

\item {} 
\textbf{lon} -- longitude (degrees E)

\item {} 
\textbf{MSE} -- moist static energy (J)

\item {} 
\textbf{N2} -- squared Brunt-Vaisala frequency (Hz\textasciicircum{}2)

\item {} 
\textbf{omega} -- vertical velocity expressed as tendency of pressure (Pa/s)

\item {} 
\textbf{p} -- pressure (Pa)

\item {} 
\textbf{Phi} -- geopotential (m\textasciicircum{}2/s\textasciicircum{}2)

\item {} 
\textbf{plcl} -- pressure at lifting condensation level (Pa)

\item {} 
\textbf{qi} -- specific humidity with respect to ice (kg/kg)

\item {} 
\textbf{ql} -- specific humidity with respect to liquid water (kg/kg)

\item {} 
\textbf{qt} -- specific humidity with respect to total water (kg/kg)

\item {} 
\textbf{qv} -- specific humidity (kg/kg)

\item {} 
\textbf{qvs} -- saturation specific humidity (kg/kg)

\item {} 
\textbf{RB} -- bulk Richardson number (unitless)

\item {} 
\textbf{RH} -- relative humidity (\%)

\item {} 
\textbf{rho} -- density (kg/m\textasciicircum{}3)

\item {} 
\textbf{ri} -- ice mixing ratio (kg/kg)

\item {} 
\textbf{rl} -- liquid water mixing ratio (kg/kg)

\item {} 
\textbf{rt} -- total water mixing ratio (kg/kg)

\item {} 
\textbf{rv} -- water vapor mixing ratio (kg/kg)

\item {} 
\textbf{rvs} -- saturation water vapor mixing ratio (kg/kg)

\item {} 
\textbf{T} -- temperature (K)

\item {} 
\textbf{Td} -- dewpoint temperature (K)

\item {} 
\textbf{theta} -- potential temperature (K)

\item {} 
\textbf{thetae} -- equivalent temperature (K)

\item {} 
\textbf{thetaes} -- saturation equivalent temperature (K)

\item {} 
\textbf{Tlcl} -- temperature at lifting condensation level (K)

\item {} 
\textbf{Tv} -- virtual temperature (K)

\item {} 
\textbf{Tw} -- wet bulb temperature (K)

\item {} 
\textbf{u} -- eastward zonal wind velocity (m/s)

\item {} 
\textbf{v} -- northward meridional wind velocity (m/s)

\item {} 
\textbf{w} -- vertical velocity (m/s)

\item {} 
\textbf{x} -- x (m)

\item {} 
\textbf{y} -- y (m)

\item {} 
\textbf{Z} -- geopotential height (m)

\item {} 
\textbf{z} -- height (m)

\end{itemize}

\textbf{Assumptions}

Default assumptions are `ideal gas', `hydrostatic', `constant g', `constant Lv',
`constant Cp', `no liquid water', `no ice', `bolton'.

\textbf{Assumption descriptions}
\begin{itemize}
\item {} 
\textbf{bolton} -- the assumptions in Bolton (1980) hold

\item {} 
\textbf{constant Cp} -- Cp is constant and equal to Cp for dry air at 0C

\item {} 
\textbf{constant g} -- g is constant

\item {} 
\textbf{constant Lv} -- latent heat of vaporization of water is constant

\item {} 
\textbf{frozen bulb} -- the bulb is frozen

\item {} 
\textbf{goff-gratch} -- the Goff-Gratch equation for es

\item {} 
\textbf{hydrostatic} -- hydrostatic balance

\item {} 
\textbf{ideal gas} -- the ideal gas law holds

\item {} 
\textbf{low water vapor} -- terms that are second-order in moisture quantities can be neglected (eg. qv == rv)

\item {} 
\textbf{no ice} -- ice can be neglected

\item {} 
\textbf{no liquid water} -- liquid water can be neglected

\item {} 
\textbf{Tv equals T} -- the virtual temperature correction can be neglected

\item {} 
\textbf{unfrozen bulb} -- the bulb is not frozen

\end{itemize}
\paragraph{Examples}

Calculating pressure from virtual temperature and density:

\begin{Verbatim}[commandchars=\\\{\}]
\PYG{g+gp}{\PYGZgt{}\PYGZgt{}\PYGZgt{} }\PYG{n}{calculate}\PYG{p}{(}\PYG{l+s}{\PYGZsq{}}\PYG{l+s}{p}\PYG{l+s}{\PYGZsq{}}\PYG{p}{,} \PYG{n}{Tv}\PYG{o}{=}\PYG{l+m+mf}{273.}\PYG{p}{,} \PYG{n}{rho}\PYG{o}{=}\PYG{l+m+mf}{1.27}\PYG{p}{)}
\PYG{g+go}{99519.638400000011}
\end{Verbatim}

Same calculation, but also returning a list of functions used:

\begin{Verbatim}[commandchars=\\\{\}]
\PYG{g+gp}{\PYGZgt{}\PYGZgt{}\PYGZgt{} }\PYG{n}{p}\PYG{p}{,} \PYG{n}{funcs} \PYG{o}{=} \PYG{n}{calculate}\PYG{p}{(}\PYG{l+s}{\PYGZsq{}}\PYG{l+s}{p}\PYG{l+s}{\PYGZsq{}}\PYG{p}{,} \PYG{n}{Tv}\PYG{o}{=}\PYG{l+m+mf}{273.}\PYG{p}{,} \PYG{n}{rho}\PYG{o}{=}\PYG{l+m+mf}{1.27}\PYG{p}{,} \PYG{n}{debug}\PYG{o}{=}\PYG{n+nb+bp}{True}\PYG{p}{)}
\PYG{g+gp}{\PYGZgt{}\PYGZgt{}\PYGZgt{} }\PYG{n}{funcs}
\PYG{g+go}{(\PYGZlt{}function atmos.equations.p\PYGZus{}from\PYGZus{}rho\PYGZus{}Tv\PYGZus{}ideal\PYGZus{}gas\PYGZgt{},)}
\end{Verbatim}

Same calculation with temperature instead, ignoring virtual temperature
correction:

\begin{Verbatim}[commandchars=\\\{\}]
\PYG{g+gp}{\PYGZgt{}\PYGZgt{}\PYGZgt{} }\PYG{n}{calculate}\PYG{p}{(}\PYG{l+s}{\PYGZsq{}}\PYG{l+s}{p}\PYG{l+s}{\PYGZsq{}}\PYG{p}{,} \PYG{n}{T}\PYG{o}{=}\PYG{l+m+mf}{273.}\PYG{p}{,} \PYG{n}{rho}\PYG{o}{=}\PYG{l+m+mf}{1.27}\PYG{p}{,} \PYG{n}{add\PYGZus{}assumptions}\PYG{o}{=}\PYG{p}{(}\PYG{l+s}{\PYGZsq{}}\PYG{l+s}{Tv equals T}\PYG{l+s}{\PYGZsq{}}\PYG{p}{,}\PYG{p}{)}\PYG{p}{)}
\PYG{g+go}{99519.638400000011}
\end{Verbatim}

\end{fulllineitems}

\index{get\_calculatable\_quantities() (in module atmos.solve)}

\begin{fulllineitems}
\phantomsection\label{atmos:atmos.solve.get_calculatable_quantities}\pysiglinewithargsret{\code{atmos.solve.}\bfcode{get\_calculatable\_quantities}}{\emph{inputs}, \emph{methods}}{}
Given an interable of input quantity names and a methods dictionary,
returns a list of output quantities that can be calculated.

\end{fulllineitems}



\section{atmos.util module}
\label{atmos:module-atmos.util}\label{atmos:atmos-util-module}\index{atmos.util (module)}
Created on Fri Mar 27 13:11:26 2015

@author: mcgibbon
\index{area\_poly\_sphere() (in module atmos.util)}

\begin{fulllineitems}
\phantomsection\label{atmos:atmos.util.area_poly_sphere}\pysiglinewithargsret{\code{atmos.util.}\bfcode{area\_poly\_sphere}}{\emph{lat}, \emph{lon}, \emph{r\_sphere}}{}
Calculates the area enclosed by an arbitrary polygon on the sphere.
\begin{quote}\begin{description}
\item[{Parameters}] \leavevmode\begin{itemize}
\item {} 
\textbf{lat} (\emph{iterable}) -- The latitudes, in degrees, of the vertex locations of the polygon, in
clockwise order.

\item {} 
\textbf{lon} (\emph{iterable}) -- The longitudes, in degrees, of the vertex locations of the polygon, in
clockwise order.

\end{itemize}

\item[{Returns}] \leavevmode
\textbf{area} --
The desired spherical area in the same units as r\_sphere.

\item[{Return type}] \leavevmode
float

\end{description}\end{quote}
\paragraph{Notes}

This function assumes the vertices form a valid polygon (edges do not
intersect each other).

\textbf{References}

Computing the Area of a Spherical Polygon of Arbitrary Shape
Bevis and Cambareri (1987)
Mathematical Geology, vol.19, Issue 4, pp 335-346

\end{fulllineitems}

\index{assumption\_list\_string() (in module atmos.util)}

\begin{fulllineitems}
\phantomsection\label{atmos:atmos.util.assumption_list_string}\pysiglinewithargsret{\code{atmos.util.}\bfcode{assumption\_list\_string}}{\emph{assumptions}, \emph{assumption\_dict}}{}
Takes in a list of short forms of assumptions and an assumption
dictionary, and returns a ``list'' form of the long form of the
assumptions.
\begin{quote}\begin{description}
\item[{Raises}] \leavevmode
\code{ValueError} --
if one of the assumptions is not in assumption\_dict.

\end{description}\end{quote}

\end{fulllineitems}

\index{closest\_val() (in module atmos.util)}

\begin{fulllineitems}
\phantomsection\label{atmos:atmos.util.closest_val}\pysiglinewithargsret{\code{atmos.util.}\bfcode{closest\_val}}{\emph{x}, \emph{L}}{}
Finds the index value in an iterable closest to a desired value.
\begin{quote}\begin{description}
\item[{Parameters}] \leavevmode\begin{itemize}
\item {} 
\textbf{x} (\emph{object}) -- The desired value.

\item {} 
\textbf{L} (\emph{iterable}) -- The iterable in which to search for the desired value.

\end{itemize}

\item[{Returns}] \leavevmode
\textbf{index} --
The index of the closest value to x in L.

\item[{Return type}] \leavevmode
int

\end{description}\end{quote}
\paragraph{Notes}

Assumes x and the entries of L are of comparable types.
\begin{quote}\begin{description}
\item[{Raises}] \leavevmode
\code{ValueError} --
if L is empty

\end{description}\end{quote}

\end{fulllineitems}

\index{d\_x() (in module atmos.util)}

\begin{fulllineitems}
\phantomsection\label{atmos:atmos.util.d_x}\pysiglinewithargsret{\code{atmos.util.}\bfcode{d\_x}}{\emph{data}, \emph{axis}, \emph{boundary=u'forward-backward'}}{}
Calculates a second-order centered finite difference of data along the
specified axis.
\begin{quote}\begin{description}
\item[{Parameters}] \leavevmode\begin{itemize}
\item {} 
\textbf{data} (\emph{ndarray}) -- Data on which we are taking a derivative.

\item {} 
\textbf{axis} (\emph{int}) -- Index of the data array on which to take the difference.

\item {} 
\textbf{boundary} (\emph{string, optional}) -- Boundary condition. If `periodic', assume periodic boundary condition
for centered difference. If `forward-backward', take first-order
forward or backward derivatives at boundary.

\end{itemize}

\item[{Returns}] \leavevmode
\textbf{derivative} --
Derivative of the data along the specified axis.

\item[{Return type}] \leavevmode
ndarray

\item[{Raises}] \leavevmode
\code{ValueError} --
If an invalid boundary condition choice is given, if both dx and x are
specified, if axis is out of the valid range for the shape of the data,
or if x is specified and axis\_x is out of the valid range for the shape
of x.

\end{description}\end{quote}

\end{fulllineitems}

\index{ddx() (in module atmos.util)}

\begin{fulllineitems}
\phantomsection\label{atmos:atmos.util.ddx}\pysiglinewithargsret{\code{atmos.util.}\bfcode{ddx}}{\emph{data}, \emph{axis=0}, \emph{dx=None}, \emph{x=None}, \emph{axis\_x=0}, \emph{boundary=u'forward-backward'}}{}
Calculates a second-order centered finite difference derivative of data
along the specified axis.
\begin{quote}\begin{description}
\item[{Parameters}] \leavevmode\begin{itemize}
\item {} 
\textbf{data} (\emph{ndarray}) -- Data on which we are taking a derivative.

\item {} 
\textbf{axis} (\emph{int}) -- Index of the data array on which to take the derivative.

\item {} 
\textbf{dx} (\emph{float, optional}) -- Constant grid spacing value. Will assume constant grid spacing if
given. May not be used with argument x. Default value is 1 unless
x is given.

\item {} 
\textbf{x} (\emph{ndarray, optional}) -- Values of the axis along which we are taking a derivative to allow
variable grid spacing. May not be given with argument dx.

\item {} 
\textbf{axis\_x} (\emph{int, optional}) -- Index of the x array on which to take the derivative. Does nothing if
x is not given as an argument.

\item {} 
\textbf{boundary} (\emph{string, optional}) -- Boundary condition. If `periodic', assume periodic boundary condition
for centered difference. If `forward-backward', take first-order
forward or backward derivatives at boundary.

\end{itemize}

\item[{Returns}] \leavevmode
\textbf{derivative} --
Derivative of the data along the specified axis.

\item[{Return type}] \leavevmode
ndarray

\item[{Raises}] \leavevmode
\code{ValueError} --
If an invalid boundary condition choice is given, if both dx and x are
specified, if axis is out of the valid range for the shape of the data,
or if x is specified and axis\_x is out of the valid range for the shape
of x.

\end{description}\end{quote}

\end{fulllineitems}

\index{doc\_paragraph() (in module atmos.util)}

\begin{fulllineitems}
\phantomsection\label{atmos:atmos.util.doc_paragraph}\pysiglinewithargsret{\code{atmos.util.}\bfcode{doc\_paragraph}}{\emph{s}, \emph{indent=0}}{}
Takes in a string without wrapping corresponding to a paragraph,
and returns a version of that string wrapped to be at most 80
characters in length on each line.
If indent is given, ensures each line is indented to that number
of spaces.

\end{fulllineitems}

\index{dpres\_hybrid() (in module atmos.util)}

\begin{fulllineitems}
\phantomsection\label{atmos:atmos.util.dpres_hybrid}\pysiglinewithargsret{\code{atmos.util.}\bfcode{dpres\_hybrid}}{\emph{p\_sfc}, \emph{hybrid\_a}, \emph{hybrid\_b}, \emph{p0=100000.0}, \emph{vertical\_axis=None}}{}
Calculates the pressure layer thicknesses of a hybrid coordinate system.
\begin{quote}\begin{description}
\item[{Parameters}] \leavevmode\begin{itemize}
\item {} 
\textbf{p\_sfc} (\emph{ndarray}) -- An array with surface pressure data.

\item {} 
\textbf{hybrid\_a} (\emph{ndarray}) -- A one-dimensional array equal to the hybrid A coefficients. Usually,
the ``interface'' coefficients are input.

\item {} 
\textbf{hybrid\_b} (\emph{ndarray}) -- A one-dimensional array equal to the hybrid B coefficients. Usually,
the ``interface'' coefficients are input.

\item {} 
\textbf{p0} (\emph{float, optional}) -- A scalar value equal to the surface reference pressure. Must have the
same units as ps. By default, 10\textasciicircum{}5 Pa is used.

\item {} 
\textbf{vertical\_axis} (\emph{int, optional}) -- The index of the returned array that should correspond to the vertical.
Must be between 0 and the number of axes in p\_sfc (inclusive).

\end{itemize}

\item[{Returns}] \leavevmode
\textbf{dpres} --
An array specifying the pressure layer thicknesses. If p\_sfc is a
float, will be one-dimensional. Otherwise, will have one more dimension
than p\_sfc. Which axis corresponds to the vertical can be given by the
keyword argument vertical\_axis. If it is not given, the vertical axis
will be 0 if p\_sfc has 1 or 2 dimensions, or 1 if p\_sfc has more
dimensions. Note that this replicates NCL behavior for 2- and
3-dimensional arrays. The size of the vertical dimension will be the
one less than the size of hybrid\_a.

\item[{Return type}] \leavevmode
ndarray

\end{description}\end{quote}
\paragraph{Notes}

Calculates the layer pressure thickness of a hybrid coordinate system. At
each grid point the sum of the pressure thicknesses equates to {[}psfc-ptop{]}.
At each grid point, the returned values above ptop and below psfc will be
set to fill\_value. If ptop or psfc is between plev levels then the layer
thickness is modifed accordingly.
\begin{quote}\begin{description}
\item[{Raises}] \leavevmode
\code{ValueError} --
If vertical\_axis is given and is not between 0 and the number of
axes in p\_sfc.

\end{description}\end{quote}

\end{fulllineitems}

\index{dpres\_isobaric() (in module atmos.util)}

\begin{fulllineitems}
\phantomsection\label{atmos:atmos.util.dpres_isobaric}\pysiglinewithargsret{\code{atmos.util.}\bfcode{dpres\_isobaric}}{\emph{p\_lev}, \emph{p\_sfc}, \emph{p\_top}, \emph{vertical\_axis=None}, \emph{fill\_value=nan}}{}
Calculates the pressure layer thicknesses of a constant pressure level
coordinate system.
\begin{quote}\begin{description}
\item[{Parameters}] \leavevmode\begin{itemize}
\item {} 
\textbf{p\_lev} (\emph{ndarray}) -- A one dimensional array containing the constant pressure levels. May
be in ascending or descending order.

\item {} 
\textbf{p\_sfc} (\emph{float or ndarray}) -- A scalar or an array containing the surface pressure data. Must have
the same units as p\_lev.

\item {} 
\textbf{p\_top} (\emph{float or ndarray}) -- A scalar or an array specifying the top of the column. Must have the
same units as p\_lev. If an array is given, must have the same shape
as p\_sfc.

\item {} 
\textbf{vertical\_axis} (\emph{int, optional}) -- The index of the returned array that should correspond to the vertical.
Must be between 0 and the number of axes in p\_sfc (inclusive).

\end{itemize}

\item[{Returns}] \leavevmode
\textbf{dpres} --
An array specifying the pressure layer thicknesses. If p\_sfc is a
float, will be one-dimensional. Otherwise, will have one more dimension
than p\_sfc. Which axis corresponds to the vertical can be given by the
keyword argument vertical\_axis. If it is not given, the vertical axis
will be 0 if p\_sfc has 1 or 2 dimensions, or 1 if p\_sfc has more
dimensions. Note that this replicates NCL behavior for 2- and
3-dimensional arrays. The size of the vertical dimension will be the
same as the size of p\_lev.

\item[{Return type}] \leavevmode
ndarray

\end{description}\end{quote}


\strong{See also:}

\begin{description}
\item[{{\hyperref[atmos:atmos.util.dpres_hybrid]{\code{dpres\_hybrid()}}}}] \leavevmode
Pressure layer thicknesses of a hybrid coordinate system

\end{description}


\paragraph{Notes}

Calculates the layer pressure thickness of a constant pressure level
system. At each grid point the sum of the pressure thicknesses equates to
{[}p\_sfc-p\_top{]}. At each grid point, the returned values above ptop and below
psfc will be set to fill\_value. If p\_top or p\_sfc is between p\_lev levels
then the layer thickness is modifed accordingly.
\begin{quote}\begin{description}
\item[{Raises}] \leavevmode
\code{ValueError} --
If vertical\_axis is given and is not between 0 and the number of
axes in p\_sfc.

\end{description}\end{quote}

\end{fulllineitems}

\index{gaussian\_latitude\_weights() (in module atmos.util)}

\begin{fulllineitems}
\phantomsection\label{atmos:atmos.util.gaussian_latitude_weights}\pysiglinewithargsret{\code{atmos.util.}\bfcode{gaussian\_latitude\_weights}}{\emph{nlat}}{}
Generates gaussian weights.
\begin{quote}\begin{description}
\item[{Parameters}] \leavevmode
\textbf{nlat} (\emph{int}) -- The number of latitudes desired

\item[{Returns}] \leavevmode
\textbf{weights} --
A one-dimensional array containing the gaussian weights.

\item[{Return type}] \leavevmode
ndarray

\end{description}\end{quote}

\end{fulllineitems}

\index{gaussian\_latitudes() (in module atmos.util)}

\begin{fulllineitems}
\phantomsection\label{atmos:atmos.util.gaussian_latitudes}\pysiglinewithargsret{\code{atmos.util.}\bfcode{gaussian\_latitudes}}{\emph{nlat}}{}
Generates gaussian latitudes.
\begin{quote}\begin{description}
\item[{Parameters}] \leavevmode
\textbf{nlat} (\emph{int}) -- The number of latitudes desired

\item[{Returns}] \leavevmode
\textbf{lat} --
A one-dimensional array containing the gaussian latitudes.

\item[{Return type}] \leavevmode
ndarray

\end{description}\end{quote}

\end{fulllineitems}

\index{geopotential\_height\_hybrid() (in module atmos.util)}

\begin{fulllineitems}
\phantomsection\label{atmos:atmos.util.geopotential_height_hybrid}\pysiglinewithargsret{\code{atmos.util.}\bfcode{geopotential\_height\_hybrid}}{\emph{psfc}, \emph{Phisfc}, \emph{Tv}, \emph{hyam}, \emph{hybm}, \emph{hyai}, \emph{hybi}, \emph{p0=100000.0}, \emph{vertical\_axis=None}}{}
Computes geopotential height in hybrid coordinates.
\begin{quote}\begin{description}
\item[{Parameters}] \leavevmode\begin{itemize}
\item {} 
\textbf{psfc} (\emph{ndarray}) -- Surface pressure in Pa.

\item {} 
\textbf{Phisfc} (\emph{ndarray}) -- Surface geopotential in m\textasciicircum{}2/s\textasciicircum{}2. If it is not the same shape as ps,
then it must correspond to the rightmost dimensions of ps. May not have
more dimensions than ps.

\item {} 
\textbf{Tv} (\emph{ndarray}) -- Virtual temperature in K, ordered top-to-bottom.

\item {} 
\textbf{hyam} (\emph{ndarray}) -- One-dimensional array of hybrid A coefficients (layer midpoints),
ordered bottom-to-top.

\item {} 
\textbf{hybm} (\emph{ndarray}) -- One-dimensional array of hybrid B coefficients (layer midpoints),
ordered bottom-to-top.

\item {} 
\textbf{hyai} (\emph{ndarray}) -- One-dimensional array of hybrid A coefficients (layer interfaces),
ordered bottom-to-top.

\item {} 
\textbf{hybi} (\emph{ndarray}) -- One-dimensional array of hybrid B coefficients (layer interfaces),
ordered bottom-to-top.

\item {} 
\textbf{vertical\_axis} (\emph{int, optional}) -- The index of Tv that corresponds to the vertical. By default, is 0 if
Tv has 3 or fewer axes, and 1 if Tv has more axes.

\end{itemize}

\item[{Returns}] \leavevmode
\textbf{Phi} --
Geopotential height values. Array has the same shape as Tv.

\item[{Return type}] \leavevmode
ndarray

\end{description}\end{quote}
\paragraph{Notes}

Assumes no missing values in input.

\end{fulllineitems}

\index{hybrid\_interpolate() (in module atmos.util)}

\begin{fulllineitems}
\phantomsection\label{atmos:atmos.util.hybrid_interpolate}\pysiglinewithargsret{\code{atmos.util.}\bfcode{hybrid\_interpolate}}{\emph{data}, \emph{ps}, \emph{hybrid\_a\_in}, \emph{hybrid\_b\_in}, \emph{hybrid\_a\_out}, \emph{hybrid\_b\_out}, \emph{p0=100000.0}, \emph{vertical\_axis=None}, \emph{extrapolate=u'missing'}}{}
Interpolates from data on one set of hybrid levels to another set of hybrid
levels.
\begin{quote}\begin{description}
\item[{Parameters}] \leavevmode\begin{itemize}
\item {} 
\textbf{data} (\emph{ndarray}) -- Data to be interpolated.

\item {} 
\textbf{ps} (\emph{ndarray}) -- Surface pressure. If given in units other than Pa, p0 should be
specified. Its rightmost axes must correspond to the rightmost axes
of data, not including the vertical axis of data.

\item {} 
\textbf{hybrid\_a\_in} (\emph{ndarray}) -- Hybrid A coefficients associated with the input data.

\item {} 
\textbf{hybrid\_b\_in} (\emph{ndarray}) -- Hybrid B coefficients associated with the input data.

\item {} 
\textbf{hybrid\_a\_out} (\emph{ndarray}) -- Hybrid A coefficients of the returned data.

\item {} 
\textbf{hybrid\_b\_out} (\emph{ndarray}) -- Hybrid B coefficients of the returned data.

\item {} 
\textbf{p0} (\emph{float}) -- Surface reference pressure. Must be in the same units as ps.

\item {} 
\textbf{vertical\_axis} (\emph{int, optional}) -- The index of data that corresponds to the vertical. By default, is 0 if
data has 3 or fewer axes, and 1 if data has more axes.

\item {} 
\textbf{extrapolate} (\emph{str, optional}) -- Determines how output values outside of the range of the input axis
should be handled. If `missing', they are set to NaN. If `nearest',
they are set to the nearest input value.

\end{itemize}

\item[{Returns}] \leavevmode
\textbf{data\_out} --
data interpolated to the new hybrid vertical axis.

\item[{Return type}] \leavevmode
ndarray

\end{description}\end{quote}

\end{fulllineitems}

\index{isobaric\_to\_hybrid() (in module atmos.util)}

\begin{fulllineitems}
\phantomsection\label{atmos:atmos.util.isobaric_to_hybrid}\pysiglinewithargsret{\code{atmos.util.}\bfcode{isobaric\_to\_hybrid}}{\emph{data}, \emph{p}, \emph{ps}, \emph{hybrid\_a}, \emph{hybrid\_b}, \emph{p0=100000.0}, \emph{vertical\_axis=None}, \emph{extrapolate=u'missing'}}{}
Interpolates data on constant pressure levels to hybrid levels.
\begin{quote}\begin{description}
\item[{Parameters}] \leavevmode\begin{itemize}
\item {} 
\textbf{data} (\emph{ndarray}) -- Data to be interpolated.

\item {} 
\textbf{p} (\emph{ndarray}) -- A one-dimensional array with the pressure levels of data. Must have
the same units as ps and p0.

\item {} 
\textbf{ps} (\emph{ndarray}) -- Surface pressure. If given in units other than Pa, p0 should be
specified. Its rightmost axes must correspond to the rightmost axes
of data, not including the vertical axis of data.

\item {} 
\textbf{hybrid\_a} (\emph{ndarray}) -- Hybrid A coefficients of the returned data.

\item {} 
\textbf{hybrid\_b} (\emph{ndarray}) -- Hybrid B coefficients of the returned data.

\item {} 
\textbf{p0} (\emph{float}) -- Surface reference pressure. Must be in the same units as ps.

\item {} 
\textbf{vertical\_axis} (\emph{int, optional}) -- The index of data that corresponds to the vertical. By default, is 0 if
data has 3 or fewer axes, and 1 if Tv has more axes.

\item {} 
\textbf{extrapolate} (\emph{str, optional}) -- Determines how output values outside of the range of the input axis
should be handled. If `missing', they are set to NaN. If `nearest',
they are set to the nearest input value.

\end{itemize}

\item[{Returns}] \leavevmode
\textbf{data\_out} --
data interpolated to the hybrid vertical axis.

\item[{Return type}] \leavevmode
ndarray

\end{description}\end{quote}

\end{fulllineitems}

\index{landsea\_mask() (in module atmos.util)}

\begin{fulllineitems}
\phantomsection\label{atmos:atmos.util.landsea_mask}\pysiglinewithargsret{\code{atmos.util.}\bfcode{landsea\_mask}}{\emph{lat}, \emph{lon}, \emph{basemap=None}, \emph{basemap\_lat=None}, \emph{basemap\_lon=None}}{}
Calculates a land sea mask for a given latitude and longitude array.
\begin{quote}\begin{description}
\item[{Parameters}] \leavevmode\begin{itemize}
\item {} 
\textbf{lat} (\emph{ndarray}) -- Latitudes in degrees N.

\item {} 
\textbf{lon} (\emph{ndarray}) -- Longitudes in degrees E.

\item {} 
\textbf{basemap} (\emph{ndarray, optional}) -- A 2D array of type byte or int that contains the baseline land sea
mask from which the new land sea mask is generated. By default uses
the grid distributed with NCL, available at
\href{https://www.ncl.ucar.edu/Applications/Data/cdf/landsea.nc}{https://www.ncl.ucar.edu/Applications/Data/cdf/landsea.nc}

\item {} 
\textbf{basemap\_lat} (\emph{ndarray, optional}) -- The latitudes of basemap. If not specified, assumes a regularly spaced
grid from -90 to 90 degrees.

\item {} 
\textbf{basemap\_lon} (\emph{ndarray, optional}) -- The longitudes of basemap. If not specified, assumes a regularly spaced
grid from -180 to 180 degrees.

\end{itemize}

\end{description}\end{quote}

\end{fulllineitems}

\index{parse\_derivative\_string() (in module atmos.util)}

\begin{fulllineitems}
\phantomsection\label{atmos:atmos.util.parse_derivative_string}\pysiglinewithargsret{\code{atmos.util.}\bfcode{parse\_derivative\_string}}{\emph{string}, \emph{quantity\_dict}}{}
Assuming the string is of the form d(var1)d(var2), returns var1, var2.
Raises ValueError if the string is not of this form, or if the vars
are not keys in the quantity\_dict, or if var2 is not a coordinate-like
variable.

\end{fulllineitems}

\index{quantity\_spec\_string() (in module atmos.util)}

\begin{fulllineitems}
\phantomsection\label{atmos:atmos.util.quantity_spec_string}\pysiglinewithargsret{\code{atmos.util.}\bfcode{quantity\_spec\_string}}{\emph{name}, \emph{quantity\_dict}}{}
Returns a quantity specification for docstrings.
\paragraph{Example}

\begin{Verbatim}[commandchars=\\\{\}]
\PYG{g+gp}{\PYGZgt{}\PYGZgt{}\PYGZgt{} }\PYG{n}{quantity\PYGZus{}spec\PYGZus{}string}\PYG{p}{(}\PYG{l+s}{\PYGZsq{}}\PYG{l+s}{Tv}\PYG{l+s}{\PYGZsq{}}\PYG{p}{)}
\PYG{g+gp}{\PYGZgt{}\PYGZgt{}\PYGZgt{} }\PYG{l+s}{\PYGZsq{}}\PYG{l+s}{Tv : float or ndarray}
\PYG{g+go}{    Data for virtual temperature.\PYGZsq{}}
\end{Verbatim}

\end{fulllineitems}

\index{quantity\_string() (in module atmos.util)}

\begin{fulllineitems}
\phantomsection\label{atmos:atmos.util.quantity_string}\pysiglinewithargsret{\code{atmos.util.}\bfcode{quantity\_string}}{\emph{name}, \emph{quantity\_dict}}{}
Takes in an abbreviation for a quantity and a quantity dictionary,
and returns a more descriptive string of the quantity as ``name (units).''
Raises ValueError if the name is not in quantity\_dict

\end{fulllineitems}

\index{strings\_to\_list\_string() (in module atmos.util)}

\begin{fulllineitems}
\phantomsection\label{atmos:atmos.util.strings_to_list_string}\pysiglinewithargsret{\code{atmos.util.}\bfcode{strings\_to\_list\_string}}{\emph{strings}}{}
Takes a list of strings presumably containing words and phrases,
and returns a ``list'' form of those strings, like:

\begin{Verbatim}[commandchars=\\\{\}]
\PYG{g+gp}{\PYGZgt{}\PYGZgt{}\PYGZgt{} }\PYG{n}{strings\PYGZus{}to\PYGZus{}list\PYGZus{}string}\PYG{p}{(}\PYG{p}{(}\PYG{l+s}{\PYGZsq{}}\PYG{l+s}{cats}\PYG{l+s}{\PYGZsq{}}\PYG{p}{,} \PYG{l+s}{\PYGZsq{}}\PYG{l+s}{dogs}\PYG{l+s}{\PYGZsq{}}\PYG{p}{)}\PYG{p}{)}
\PYG{g+gp}{\PYGZgt{}\PYGZgt{}\PYGZgt{} }\PYG{l+s}{\PYGZsq{}}\PYG{l+s}{cats and dogs}\PYG{l+s}{\PYGZsq{}}
\end{Verbatim}

or

\begin{Verbatim}[commandchars=\\\{\}]
\PYG{g+gp}{\PYGZgt{}\PYGZgt{}\PYGZgt{} }\PYG{n}{strings\PYGZus{}to\PYGZus{}list\PYGZus{}string}\PYG{p}{(}\PYG{p}{(}\PYG{l+s}{\PYGZsq{}}\PYG{l+s}{pizza}\PYG{l+s}{\PYGZsq{}}\PYG{p}{,} \PYG{l+s}{\PYGZsq{}}\PYG{l+s}{pop}\PYG{l+s}{\PYGZsq{}}\PYG{p}{,} \PYG{l+s}{\PYGZsq{}}\PYG{l+s}{chips}\PYG{l+s}{\PYGZsq{}}\PYG{p}{)}\PYG{p}{)}
\PYG{g+gp}{\PYGZgt{}\PYGZgt{}\PYGZgt{} }\PYG{l+s}{\PYGZsq{}}\PYG{l+s}{pizza, pop, and chips}\PYG{l+s}{\PYGZsq{}}
\end{Verbatim}

Raises ValueError if strings is empty.

\end{fulllineitems}



\chapter{Indices and tables}
\label{index:indices-and-tables}\begin{itemize}
\item {} 
\emph{genindex}

\item {} 
\emph{modindex}

\item {} 
\emph{search}

\end{itemize}


\renewcommand{\indexname}{Python Module Index}
\begin{theindex}
\def\bigletter#1{{\Large\sffamily#1}\nopagebreak\vspace{1mm}}
\bigletter{a}
\item {\texttt{atmos}}, \pageref{atmos:module-atmos}
\item {\texttt{atmos.constants}}, \pageref{atmos:module-atmos.constants}
\item {\texttt{atmos.decorators}}, \pageref{atmos:module-atmos.decorators}
\item {\texttt{atmos.equations}}, \pageref{atmos:module-atmos.equations}
\item {\texttt{atmos.solve}}, \pageref{atmos:module-atmos.solve}
\item {\texttt{atmos.util}}, \pageref{atmos:module-atmos.util}
\end{theindex}

\renewcommand{\indexname}{Index}
\printindex
\end{document}
